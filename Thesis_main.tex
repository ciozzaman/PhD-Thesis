\documentclass[]{yorkThesis}  % Define class
\usepackage[utf8]{inputenc}	% Input encoding
\usepackage{amsmath}	% Maths symbols
\usepackage{amsfonts}	% Maths fonts
\usepackage{amssymb}	% More maths stuff
\usepackage{bm}	% bold math symbols
\usepackage{graphicx}	% Allows for embedded graphics
%\usepackage{graphics}   % to automatically resize a table
\usepackage{epstopdf}
\usepackage[compress]{cite}	% Allows for the use of a bibliography, and automatically handles things like numbering
\usepackage{tabu}


\def\chapterautorefname{Chapter}    % have Section as capital
\def\sectionautorefname{Section}    % have Section as capital
\def\subsectionautorefname{Section}    % have Section as capital
\def\subsubsectionautorefname{Section}    % have Section as capital

\usepackage{float}
\usepackage{xcolor}   % allows for highlighted text
\usepackage{soul}   % allows for highlighted text
\usepackage{caption}   % allows subfigures
\usepackage{subcaption}   % allows subfigures
%links


\DeclareUnicodeCharacter{2060}{!!!!I-AM-HERE!!!!!}   % this is to make evident if there is some weird unicode character I need to remove
\DeclareUnicodeCharacter{03BD}{!!!!I-AM-HERE!!!!!}   % this is to make evident if there is some weird unicode character I need to remove
\DeclareUnicodeCharacter{03C3}{!!!!I-AM-HERE!!!!!}   % this is to make evident if there is some weird unicode character I need to remove


\usepackage{gensymb}    % this is to have the degree symbol



% \usepackage[hidelinks]{hyperref}	% Allows for embedded clickable
\usepackage[hidelinks,pagebackref]{hyperref}	% Allows for embedded clickable and back reference at the page where the reference was used


% Title
\title{Study of detachment and the processes involved in its dynamics in MAST-U and Magnum-PSI via radiation and emission analysis}
% Author
\author{Fabio Federici, MSc.}
% Dept - required for front page. Do not put "Department of" as York do not allow it
\dept{Physics} 
% Supervisor - Not required for the document
\supervisor{Dr Bruce Lipschultz \\ Dr Matthew L Reinke}
% Date - leave blank to put todays date, or write in a specific date.  This should be the month and year of first submission
\submitdate{January 2023}
% Linespacing - 1.25 is somewhat conservative, most would opt for 1.5
\linespread{1.5}



% This is where the actual content is created
\begin{document}

% Create title page (uses spacing fonts outlined in the class file yorkThesis.cls)
\titlePage
% Start the abstract page
\abstract
% Add abstract to the list of contents (university requirement)
\phantomsection
\addcontentsline{toc}{chapter}{Abstract}
% Text for abstract goes here
The process of divertor detachment in tokamaks is associated with the reduction of power and particles to the targets. This is deemed necessary to limit damage and erosion of solid surfaces and preserve the target.

The evolution of the radiation profile is studied in MAST-Upgrade, thanks to the novel Infra Red Video Bolometer (IRVB) diagnostic. The IRVB was optimised to observe the lower x-point and divertor and was successfully calibrated and verified post installation.

The movement of the radiation along the divertor legs and inner separatrix was compared with other metrics of detachment. In MAST-U L-mode plasmas, progress of radiative detachment happens in the same sequence as in large aspect ratio tokamaks. Inner leg detachment is gradual, appearing to be contrary to expectations from theory, which is beneficial to detachment control.

The IRVB, in combination with spectroscopic measurements, was used to infer that in MAST-U, during detachment of a super-x plasma, hydrogenic emission dominates radiation on the outer leg. This is in stark contrast with another carbon machine, TCV, where carbon dominates instead.

ELM-like pulses have been reproduced on the linear machine Magnum-PSI at DIFFER. The target chamber neutral pressure was increased to simulate detachment of the steady state plasma. In some cases the ELM-like pulse energy was completely dissipated in the volume. This can potentially translate to tokamaks, if the ionisation front has sufficiently receded from the target.

Despite significant radiative losses, most of the plasma energy losses are due to potential energy exchange. Both molecular assisted recombination and dissociation are important, with the latter being a more efficient path to dissociation than electron impact dissociation.

Studying detachment in both machines it is determined that, when the plasma temperature drops below $\sim$5eV, it is necessary to include molecular assisted reactions to accurately model the plasma’s power and particle balance.

% Create contents page
\contents

% Acknowledments section
\acknowledgments
\phantomsection


This work was only possible thanks to the support of my supervisor Bruce Lipschultz; my ``unofficial'' supervisor Matthew Reinke; CCFE exhaust group; the collaborators from CCFE and DIFFER; the staff of the YPI; the engineers and technicians of MAST-U and Magnum-PSI, and all my PhD colleagues in and out of the CDT, to all of which I'm very thankful.

First of all, I would like to thank Bruce for his support in these years. The project had its ups and downs, not helped by a pandemic in its middle, but we did the best we could to navigate towards our goals. It must have been difficult to manage a student based in a different lab, with a background in engineering rather than physics and eager to always fall in the newest rabbit hole, but he did a very good job. I'm also thankful for Bruce taking care of the organisational hiccups sometimes arising from the hardware work, providing the critical push for things to follow through.

I would like to thank Matt, as he made the entire PhD possible and was responsible for engineering and procuring the hardware, with all the headaches it implies. He taught me how to design, calibrate and use an IRVB diagnostic and his help was necessary in interfacing with a device as complicated as MAST-U. He taught me the care and thoughtfulness necessary to operate on a machine as complex as MAST-U and didn't get angry when ``magic smoke'' was accidentally released. Matt was often helpful to smooth the discussion and was always seeking tangible results.

Plenty of others have helped me during these 5 years but they would not fit on these few pages. I can explicitly mention only a few, so I apologize in advance for who is not here, but I'm still grateful to.

I thank Tom Farley, whose help was necessary to manage the multitude of interfaces between the IRVB and MAST-U and helped to deal with the documentation. This team work was vital and I wish him and his family the best. I would like to thank Gijs Akkermans for providing the first contact that made the collaboration with DIFFER possible, his teachings on spectroscopy, and allowing me to meet so many wonderful people.

I want to thank everyone that helped me progress towards my results in MAST-U, and among many in particular Andrew Thornton, Jack Lovell, Richard Martin, James Harrison, Peter Ryan, Joe Allcock, Mike Robson, Kandan Balamurugan, Nigel Thomas-Davies and all the MAST-U team. I also want to thank those who helped me navigate through the different environment of DIFFER like Ivo Classen, Thomas Morgan, Jonathan van den Berg and all the Magnum-PSI operations team.

I can't thank Kevin Verhaegh enough, whose spectroscopy and operations knowledge was instrumental, together with help from Daljeet Singh Gahle, to steer the work on molecules towards its scientific objectives. Chris Bowman was invaluable in providing support for everything Bayesian.

I'm infinitely grateful to Yoshika Terada who gave me the courage to seek and start the PhD in the first place. She supported me all this years, from Massa Lombarda to Oxford, and has always believed in me. She also helped me in developing the tools to analyse the raw and processed data.

I would like to thank all the people that made my stay in the UK a lot more fun, and I'm happy to have had the chance to meet, of which I would like to mention Omkar, Hasan, Eduardo, Steve, Charlie, Lena, Tom N, Jack, Sam, Simon, Joe, Michail and all the Fusion CDT group. I thank in particular Andrew for his support and willingness to adapt to the Italian ways; Nick for editing this thesis; Mads for introducing me to caving; Yacopo and Nicola to bring Italian vibes in this far away land. I'm thankful for all the great people I met in The Netherlands and are part in making it a \emph{beautiful} place, like Aaron, Michele, Luca, Ana, Qin, Devyani and all the Pineapples.

I'm thantful to Vittorio Colombo, who first taught me about plasmas and, many years later, helped me secure the PhD.

Last but not least I would like to thank my family and all my friends in Italy that have always supported me.

Voglio ringraziare i miei amici d'infanzia, che sono da sempre al mio fianco e che sono sempre felice di incontrare: Diego, Lorena, Bert. Voglio anche ringraziare i miei amici dell'università, con sono ancora vicino e il cui successo mi ha motivato per intraprendere il dottorato: Alberto, Viola, Benny, Fede.

Voglio anche ringraziare Massimo Foresti e tutti gli ex-colleghi dell'Elettrotecnica Imolese che, nonostante per iniziare il dottorato abbia dovulo lasciare l'azienda, mi sono sempre stati amici.

Infine voglio ringraziare la mia famiglia, che nonostante la scelta di intraprendere il dottorato più tardi della norma e di stare così lontano da casa mi ha sempre voluto bene e sostenuto.\\



\fbox{\begin{minipage}{\linewidth}%{15em}

Support for M. L. Reinke’s contributions was in part provided by Commonwealth Fusion Systems.

This work has been carried out within the framework of the EUROfusion Consortium, funded by the European Union via the Euratom Research and Training Programme (Grant Agreement No 101052200-EUROfusion). Views and opinions expressed are, however, those of the author(s) only and do not necessarily reflect those of the European Union or the European Commission. Neither the European Union nor the European Commission can be held responsible for them.

The authors wish to also acknowledge Byron J. Peterson for providing the blackened and calibrated platinum foil used as absorber in the IRVB.

This work was supported in part by the DIFFER institute.


\end{minipage}}






% Declaration - explicitly say that your thesis is your own work.  Some text is generated automatically from the class file.  Anything else can be added below
\declaration

Portions of this thesis are based on manuscript which has been published to peer-reviewed journals. I am the first author of this manuscript. Where appropriate, this work has been referenced. The vast majority of the work has been done by the main author and is explicitly stated otherwise. The manuscript is:

F. Federici, M. L. Reinke, B. Lipschultz, A. J. Thornton, J. R. Harrison, J. J. Lovell, and M. Bernert. Design and implementation of a prototype infrared video bolometer (IRVB) in MAST Upgrade. \emph{Review of Scientific Instruments}, 2023. Also cited in this work as \cite{Federici2023}.



%You can add some stuff here but you don't have to if you haven't got anything you need to specifically declare, the required input is included automatically.

% % Main Matter % %
% The general document structure is detailed by the graduate research school on the website under "Format your Thesis"
% Labels allow you to link sections of your thesis together with hyperlinks within the PDF


% Executive summary section

% \phantomsection
\chapter*{Executive summary}
\addcontentsline{toc}{chapter}{Executive summary}


The changes of radiation profile due to detachment were characterised in MAST-U. This was enabled by the new infra red video bolometer (IRVB) diagnostic. The diagnostic was originally designed by Matthew Reinke, based on previous work on NSTX-U.\cite{VanEden2016} %A thin foil is shielded by the plasma, excepted for a pinhole. The light from the plasma goes through the pinhole and is absorbed by the foil. The foil is then monitored by an infra red camera. With known foil properties, its temperature can be used to determine the power absorbed by the foil and the emissivity can be obtained via tomographic reconstruction. 
The field of view (FOV) of the IRVB is designed to accurately measure the radiated power profiles around the lower x-point region, where large spatial gradients of emissivity are expected. The FOV includes a poloidal view of the plasma covering $\sim$2/3 of the foil and a tangential view over the rest. The diagnostic was optimised using the ray tracing code CHERAB and fully characterised prior to installation. Due to restricted vessel access, the match between design and real FOV was verified using known features of the plasma itself, like the silhouette of the central column, highlighted by disruptions, and bright spots corresponding to fuelling locations. This verification is a difficult task for foil bolometers, as the minimum signal that can be detected is of brightness similar to the plasma, so they cannot be calibrated with long exposure images like other camera diagnostics.

In parallel to the hardware and calibration activities, an inversion algorithm was developed to obtain the emissivity distribution. This follows a Bayesian approach where, given a set of priors (limited negative emissivity and regularity), the uncertainties of the components of the IRVB are propagated so as to find the most likely radiation distribution that yields the measurements. The radiated power inferred has a spatial resolution of the order of few cm, difficult to achieve in the x-point and divertor region, and unprecedented in spherical tokamaks. This demonstrates the viability of the IRVB to measure radiation in a region other than the core, for which they are usually employed.

In MAST-U the IRVB was used to determine that, in a conventional divertor L-mode plasma, radiative detachment follows the same progression as in large aspect ratio tokamaks: the inner target detaches first, followed by the outer, then the radiation climbs the inner separatrix to form a MARFE-like structure at the midplane that can then penetrate into the core. In MU01 no extrinsic impurities were fed to the plasma, and the radiation was not observed to penetrate the core from the x-point, as was previously observed in other devices.\cite{Wiesen2017} The radiation on the inner leg was observed with unprecedented resolution and it was determined that its movement from target to x-point is gradual, appearing to be contrary to expectations from theory.\cite{Lipschultz2016} The IRVB was also used, in conjunction with the MWI and other spectroscopic measurements, to infer that, after the particle flux roll over in a super-x plasma, the radiation is mainly dominated by hydrogenic radiation, unlike in the comparable carbon wall devices TCV. It is inferred that, after the ionisation front recedes from the target and the plasma cools, radiation from molecular effects becomes dominant.


As part of a collaboration with The DIFFER institute, ELM-like pulses were reproduced in the linear machine Magnum-PSI. The steady state plasma was progressively detached by increasing the neutral pressure in the target chamber, simulating the burn through in a tokamak divertor. It was observed that by increasing the neutral pressure it was possible to reduce the energy delivered to the target, up to complete shielding in extreme cases. This is associated with a reduction of the heat flux factor, a measure of the effect of heat transients on wall components. This observation could potentially be applicable for tokamaks, as the connection length between the radiation front and the target can be of the order of meters (in deep detachment) while the pressure can reach a few tens of Pa, as it was in Magnum-PSI. This can be further improved if the divertor is baffled, so that the neutral compression is enhanced, and an advanced divertor configuration like the super-x is adopted, resulting in a much longer connection length.

The optical emission spectroscopy (OES) setup was upgraded by the author and Gijs Akkermans in order to acquire intra-ELM data. The OES is composed by a series of parallel poloidal lines of sight, so that Balmer line emission can be determined radially. This was used, together with a crude model of the plasma column, to infer the processes in the plasma via a Bayesian inference. For this study the effect of molecular reactions was accounted for in terms of line emission and in both the particle and power balance. It is observed that even if radiative losses are significant, exchanges of potential energy dominate the plasma energy losses. Of the molecular reactions, the path to molecular assisted dissociation is the most important, as it is more efficient than electron impact dissociation. Molecular assisted recombination is also important, especially at intermediate temperatures, but electron ion recombination dominates for cold plasmas. It is found that ionisation dominates for temperatures above $\sim$5eV while molecular processes become important in colder plasmas and electron ion recombination dominates below $\sim$1.5eV.

The results from both devices indicate that in deep detachment and during the burn through, when plasmas interact with colder neutrals and molecules, the effect of molecular reactions becomes important at mid to low temperatures. This is a useful insight for the development of models used to estimate the plasma energy losses during detachment or removed from ELMs before they reach the target.


\pagestyle{headings}
\chapter{Introduction}\label{chapter1}



\graphicspath{./Chapters/appendix1/figs/}
In this section I will describe the background and the current status of the research for reliable fusion energy.
\section{Fusion energy} \label{Fusion_energy}

In a fusion reaction two light nuclei fuse and a heavier nucleus is generated. The energy released in the reaction depends on how strongly protons and neutrons are bound in the nucleus of reactant and products. This strength is expressed by the average binding energy per nucleon, that is the energy necessary to decompose one nucleus over the number of nucleons. The energy released by the nuclear reaction is given by the difference of the total binding energy in the products minus the reactant. \autoref{fig:binding_energy} shows that it is possible to obtain energy in two ways: fusion and fission. In fusion, the lighter the reactant the larger the energy released.

\begin{figure}
	\centering
	\includegraphics[width=\linewidth]{Chapters/chapter1/figs/binding energy.PNG}
	\caption{Average binding energy per nucleon \cite{YanNingNaulinVolkerWanBaonianXu2014}}
	\label{fig:binding_energy}
\end{figure}

Another factor to consider in comparing different fusion reactions is the likelihood of the reaction to happen. This is characterised by the fusion cross section, that depends on the kinetic energy of the incident particle. Considering an ensemble of particles this kinetic energy translate in temperature. The lower the temperature for which the cross section reaches its peak the easier to achieve fusion.
For this and other reasons the reaction that is the main focus of current research is the fusion of deuterium and tritium to helium: \cite{Miley1974}

\begin{equation}
{ }^2_1 D+ { }^3_1T \leftarrow { }^4_2He+{ }^1_0n
\label{eq:fuse}
\end{equation}

In this reaction are released 3.5MeV in kinetic energy of the alpha particle and 14.1MeV in the neutron. Tritium is not available in nature because it is radioactive with a short half life, so it must be produced. It is foreseen to produce it using the neutron released in reaction \ref{eq:fuse} in the reactions \ref{eq:fuse1} and \ref{eq:fuse2}

\begin{equation}
{}^{6}Li + n \leftarrow {}^{4}He +T +4.78MeV
\label{eq:fuse1}
\end{equation}

\begin{equation}
{}^{7}Li + n \leftarrow {}^{4}He +T +n +2.47MeV
\label{eq:fuse2}
\end{equation}

The initial fuels in this cycle then are deuterium and lithium, both widely available in nature.

\section{Plasma confinement}
\hl{triple product, H mode, SOL, ELMs}

The temperatures that are relevant for fusion application are measured in keV, that correspond to millions of °C. At these temperatures matter is in the state of plasma, meaning that atoms are ionised and nuclei and electrons can move separately. These temperatures are of the same order of magnitude of the centre of the Sun. No material can withstand them so special techniques must be adopted to confine the reactant long enough for a significant fraction to fuse. The most important figure of merit of the achieved confinement is the fusion triple product, related to Lawson criteria, that returns the minimum product of plasma density, temperature and confinement time to achieve ignition (\autoref{eq:fuse3}). Ignition is the condition when the fusion power output is sufficient to maintain a constant temperature against all losses without external heating.

\begin{equation}
{n_e} {\tau }_{E} T_e  >=  5\cdot{10}^{21}{ m }^{ -3 }skeV
\label{eq:fuse3}
\end{equation}

The most viable way to maintain this environment is expected to be
through magnetic confinement, and specifically the tokamak (Magnetic Confinement Fusion, MCF).
In MCF the fuel is confined in plasma state thanks to its behaviour when exposed to electromagnetic fields. Every electrically charged particle is subject to Lorentz force:


\begin{equation}
F = q ( E+ vxB )
\label{eq:lorentz}
\end{equation}

In the presence of a magnetic field the particle gyrate with a circular motion in the direction perpendicular to field lines and is unperturbed in the parallel direction. In a tokamak the plasma is arranged in a doughnut shape, closely surrounded two main set of coils: the toroidal field coils that generate the toroidal magnetic field and the central solenoid that with a time varying current induces a current through the plasma that generates the poloidal magnetic field. The toroidal geometry cause drifts in the plasma that would cause it to quickly reach the walls and cool down. To stabilise the plasma in its doughnut shape other coils have to be added to drive a current in it and to fix its radial position and shape it as requested.\cite{Chen1974} The magnetic field generated in this way is composed of concentric flux surfaces that expand to the coils. The final configuration is outlined in \autoref{fig:mcf}. Shown in pink is a magnetic surface, that is the set of all the positions a particle can assume when streaming unimpeded following magnetic field line. JET is currently the largest tokamak and has achieved triple products higher than $10^{21} m^{-3}skeV$ in 1997 \cite{Gormezano1998} and recently the record for fusion energy produced of 59MJ. \cite{Gibney2022}

\begin{figure}
	\centering
	\includegraphics[width=\linewidth]{Chapters/chapter1/figs/mcf.png}
	\caption{Schematic of the main components of a tokamak \cite{CulhamCentreforFusionEnergy2018}}
	\label{fig:mcf}
\end{figure}

The plasma can move freely along the field lines and slowly across it, so in time it drifts from the centre toward the wall via cross field transport caused by collisions and turbulence. One of the important
evolutions of the tokamak design is to add a coil parallel to the plasma with a current in the opposite direction. This changes the magnetic configuration in such a way that the core of the plasma is not any more in direct contact with the solid limiter. The first magnetic surface that crosses solid surfaces from the core is referred as Last Closed Flux Surface (LCFS) or separatrix and where the poloidal field goes to zero is the x-point. The separatrix crosses the solid surfaces in well defined regions, called divertor targets. The plasma that escapes from the core reaches the separatrix and then follows the magnetic field around the core and reaches the targets. In this way the core is shielded from the impurities of the plasma/surface interaction and can efficiently expel the products of the fusion. This motion is much faster than cross field one and it is associated with a layer out of the LCFS called Scrape-Off Layer (SOL). The time a particle needs to cross the plasma from the centre and reach the wall defines the confinement time.

\section{H-mode}
A way to limit cross field transport in the core region and move towards ignition is to operate in the so called H mode. This regime is obtained when the energy flux across the separatrix is increased over a threshold that depends on the discharge conditions.\cite{Ryter1998} The physics behind the specific value of the L-H threshold is still not fully understood, but its effect is to induce a transport barrier at the edge of the core plasma. This barrier strongly reduces the anomalous transport due to turbulence in a thin region around the core, allowing for a better energy and particle confinement there. One drawback of the H-mode is that it causes cycles of accumulation of energy in the core and sudden release as bursts of hot plasma. These so called Edge Localised Modes (ELMs) cause transient high particle and thermal flux on the target. Moreover in order to sustain the H-mode a certain amount of power must cross the separatrix, otherwise the H-L back transition can occur. This constitutes a lower bound to the heat and particles that need to be dissipated from the separatrix to the target.

\section{The exhaust problem}
\hl{thin SOL, melting/erosion}

The issue that arises with the divertor configuration is that the SOL is  very narrow, of the order of mm, and all the exhaust heat is delivered to a tiny area of the target. For ITER, a fusion device in construction in France that will be a closer step to a power production device, this heat is in the order of $GW/m^2$. The maximum power that can be removed from a surface with current technologies is $10-20MW/m^2$. \cite{Lipschultz2018} If the heat to the target is not mitigated this will pose a serious limitation on its lifetime and the amount of impurities that reaches the plasma, making impossible profitable operations. 
Sputtering is the phenomenon of emission of one or multiple atoms from a solid surface caused by interaction with an ion. There are various types of possible interactions but, as a whole, larger the energy of the ion larger number of atoms are sputtered from the surface. The sputtered atoms can end in the plasma, polluting it, of be redeposited to the target, changing its composition and properties. The sputtering can be so severe to be the main limiting factor of the lifetime of the plasma facing components.

\section{Divertor shaping}
\hl{total/poloidal flux expansion}

The heat and particle flux can be reduced by tilting the target in respect to the separatrix and by locally reducing the magnetic field, thus spreading the heat on a larger surface, but this is not enough. For power plant sized devices the heat flux due to the plasma recombination at the surface would be in itself too large.
Under investigation are different divertor designs (see some on \autoref{fig:divertor_geometry}) with the purpose, among others, to ease the thermal burden on the target

\begin{figure}
	\centering
	\includegraphics[width=\linewidth]{Chapters/chapter1/figs/divertor geometry.jpg}
	\caption{Example of different divertor configurations. Left:Standard Vertical Plate Divertor (SVPD), Super-X Divertor (SXD) and X-point Divertor (XPTD). Right: Long Vertical Leg Divertor (LVLD) \cite{Umansky2017}}
	\label{fig:divertor_geometry}
\end{figure}

The two above mentioned effects combined lead to a reduction of the heat flux density around 20 times for a tokamak of SVPD divertor type like will be on ITER, still short of another 10 times reduction needed to respect the 10-20 MWm-2 limit. The divertor configuration introduces an asymmetry in the toroidal geometry due to the two targets located at different radius. In terms of energy and particle redistribution in the SOL the inner target will receive a higher ion flux and lower energy flux compared to the outer one. As it will become clear later, on the inner target this causes stronger recycling and eases the thermal load. Therefore the outer target is normally subjected to more severe conditions. [\cite{Potzel2014} and references there therein]

\section{Radiative dissipation scenario}
\hl{energy removal --> radiative scenario, impurities}

One way to decrease the heat flux in the target is to induce radiation.
Radiation occurs naturally in plasmas and it depends greatly on temperature and density of each specie. When the atom is not yet fully ionised the electronic levels can be excited by collision and de excite emitting a photon. The atom can also recombine with free electrons and release a photon too. These energies correspond to the peaks in the curves in \autoref{fig:loss_curve}. If the temperature is so high that atoms are fully ionised the only radiative mechanism is Bremsstrahlung radiation, that is much less efficient. This corresponds to the monotonic increasing right part of the curves in \autoref{fig:loss_curve}.

\begin{figure}
	\centering
	\includegraphics[width=\linewidth]{Chapters/chapter1/figs/loss curve.png}
	\caption{Loss function data from the ADAS data base (solid lines) for the different elements \cite{Lux2015}}
	\label{fig:loss_curve}
\end{figure}

At the temperature of interest in the SOL (~ 100s of eV) hydrogen is ionised and radiates weakly. With the addition of impurities sputtered from the wall or specifically seeded in the discharge the energy can be transferred from the plasma to the impurities and then radiated. This mechanism can be exploited to dissipate a significant amount of power from the core and edge of the plasma. This regime is referred to as the radiative scenario.


\section{Detachment regimes}
\hl{particle removal -> recycling, detachment}

The heat flux to the target is the sum of the component parallel to field lines $q_{par}$, the orthogonal one $q_{ort}$ and a component given by the energy released with recombination $q_{rec}$, given by  \autoref{eq:detachment}.

\begin{equation}
\begin{split}
{ q } _{ par } = { n } _{ e,t } { c } _{ s } {  \gamma  } _{ sh }{ k } _{ B }{ T } _{ e,t } \; \alpha \; { p } _{ e,t }{{ T } _{ e,t }}^{0.5 }  \;,\; { q } _{ ort } = { q } _{ par } \left( \frac {{ B } _{ p }} {{  B  }_{ \phi  }}   \right) _{ t} \; \alpha \; { p } _{ e,t }{{ T } _{ e,t }}^{0.5 } \\ \;,\; { q } _{ rec } = { n } _{ e,t } { c } _{ s } {  E  } _{ pot } \; \alpha \; { p } _{ e,t }{{ T } _{ e,t }}^{-0.5 }
\end{split}
\label{eq:detachment}
\end{equation}

where $\gamma_{sh}$ is the heat sheath transmission factor, $n_{e,t}$ and $T_{e,t}$ are the electron density and temperature in front of the target, $c_s$ is the sound speed, $B_p$ and $B_{\Phi}$ are the poloidal and toroidal magnetic field and $E_{pot} \approx 15.8 eV$ is the potential energy released per ion when recombining to neutral deuterium molecules on the target plate. \cite{Reimold2015} One possible solution to achieve a further reduction in heat flux is to cause gradients of temperature and pressure along the field lines in the SOL. This should be done in such a way to minimize the impact on the core plasma.

A way to achieve this is to induce plasma detachment. This phenomenon has been observed in a series of tokamaks [\cite{Reimold2015} and references therein] and can be induced by impurity seeding or fuelling.
In an attached regime the plasma streams through the SOL and reaches the target. At the target the charged particles recombine and become neutrals. They are generated in an excited state and cause strong radiation where they are generated. They are not bound from magnetic fields, and can move freely until they collide with other neutrals or particles from the plasma. A neutral can then re-ionise and stream along field lines, returning to the target or entering the plasma again. This is called recycling.

If the density in the SOL is high enough the neutrals generated by the plasma streaming outward will interact mainly in the SOL and the plasma will loose part of its energy and momentum on its way out. To further increase the energy loss low Z impurities like Nitrogen can be seeded in the SOL: they will ionise only partially, with the possibility to radiate at higher temperature than hydrogen or helium. The lower the temperature the higher the energetic ionisation cost. This because lower temperature means smaller excitation by collisions  of the bound electron up to ionisation with radiation losses in the way, instead of a single higher energetic collision directly to ionisation. This cause even more radiative cooling. 

From the experimental point of view it is observed that the particle flux to the target increases for increasing plasma gas puffing. Increasing further puffing or impurity seeding the particle flux reaches a maximum and then decrease. This is called rollover and corresponds with the onset of detachment. Pushing the process forward the particle flux decrease and saturates. In the case the plasma could loose all its kinetic energy, or temperature, it still maintains the energy associated with its creation, the ionisation energy, $13eV$. In a DEMO scale device this residual energy flux would be still high enough to exceed the $10-20 MWm^{-2}$ limit on the target. \cite{Krasheninnikov2017a} To reduce it even further the temperature must drop below $1eV$ to trigger volume recombination. This is accompanied by a strong radiation increase from where recombination happens. The flux drop can continue up to the point that the measured plasma temperature at the target reaches a minimum and the radiating region recedes upstream along the field lines. This comes from the balance between the power entering the SOL with the one dissipated by radiation. It has been demonstrated that the threshold for rollover is proportional to a certain value of the ratio $ \frac {q_{rec}} {P_{up}}$ ($q_{rec}$ is the specific energy flux into hydrogen recycling region, $P_{up}$ is the upstream plasma pressure). \cite{Krasheninnikov1999}\cite{Krasheninnikov2016}


\section{analytic models}
2PM, DOD, detachment sensitivity
\section{Atomic and molecular processes}
            1. attached → hot → atomic physics
            2. detached → cold → atomic and molecular physics
\section{The x-point radiator}
\hl{extreme end of detachment before MARFE, increased radiated power}

When a discharge in H-mode is close to the detachment threshold in a tokamak with a conventional divertor configuration and the impurity density is increased 4 stages of detachment are defined as per the description in \cite{Reimold2015}:

\begin{enumerate}
    \item Onset of detachment: the inner target detaches inter-EMLs (even due to recycling and without impurity seeding), the outer target is attached
    \item Fluctuating state: radiative fluctuations in the kHz appear at the X-point, inner target detaches inter-EMLs
    \item Partial detachment at outer target: inner target always detached, outer target detached inter-ELMs, strong radiation at the X-point, fluctuations frequency decrease to ELMs scale, ELMs amplitude decrease.
    \item Complete detachment: inner and outer target always detached, sporadic ELMs, radiator moves from X-point further into the confined plasma
\end{enumerate}

For this type of discharge the radiator close to the X-point appears when detachment starts at the inner target, while it is only enhanced by the detachment on the outer target. Because the parameter of interest is the reduction of thermal flux on the outer target a feature that will always be present is a strong radiator located close to the X-point.

\section{Effect of XPR}
\hl{poloidal asymmetry, pedestal flattening, loss confinement}


The X-point radiator is a region of steep temperature gradient, where the temperature goes from the hot upstream plasma from the core to the cold region where ion / neutral interactions dominate. The presence of such a thing at or inside the separatrix can lead to confinement degradation. In this context what is referred as confinement H98 is the ratio of the actual energy confinement time $\tau_{th}$ over a reference. The reference is a confinement time from a scaling law obtained fitting the confinement time of many tokamak experiments for in a certain operating mode. For ELMy H-mode the scaling law mostly used is the ITER Physics Basis (IPB) 98(y,2). \cite{Doyle2007} The scaling given by \autoref{eq:h98}.


\begin{equation}
{ H }_{ 98 }={\tau }_{ th }/{\tau }_{ th,98y2 } \; , \; {\tau }_{ th,98y2 }=0,0562 {{ I }_{ P }}^{ 0.93} {{ B }_{ t }}^{ 0.15} {{ n }_{ 19 }}^{ 0.41} {{ P }_{ L }}^{ -0.69} {{ R }_{  }}^{ 1.97} {{ \varepsilon  }_{  }}^{ 0.58} {{ \kappa  }_{ a }}^{ 0.78} {{ M }_{  }}^{ 0.19}
\label{eq:h98}
\end{equation}


with $I_P$ plasma current, $B_t$ toroidal magnetic field, $n^{19}$ electron density in units of $10^{19} \#/m^3$, $P_L=P-dW/dt$ power loss with $P$ heating power and $W$ stored energy, $R$ major radius, $\varepsilon$ inverse aspect ratio, $\kappa _a$ elongation, $M$ ion mass number.
It is possible to calculate the energy confinement time from magnetic coils measurements and the energy transferred to the plasma as heating ($P_{heat}$). $\tau_{th}$ is defined by 

\begin{equation}
\begin{split}
\frac {dW} {dt}={P}_{heat} - \frac {W} {\tau }_{ th } \; , \; W={ 3} over{2} \langle p \rangle V \; , \; \langle p \rangle = \frac {{ \mu }_{ 0 } {{ I }_{ p }}^{ 2 } { \beta }_{ p }} { 8 {\pi}^{2} {a}^{2}  } \\ {\beta }_{ p } = \langle \frac { n {k}_{B} T} { {{B}_{p}}^{2} /(2 {\mu}_{0}) } \rangle \; , \; {P }_{ heat }={ P }_{ ohmic }+{ P }_{ NBI }+{ P }_{ RF }
\label{eq:tau}
\end{split}
\end{equation}


With $V$ plasma volume, $\langle p \rangle$ volume-averaged plasma kinetic pressure, $I_p$ plasma current, ${{ \beta }_{ p }}$ poloidal beta, $B_p$ poloidal magnetic field, $a$ minor radius, $P_{ohmic}$, $P_{NBI}$, $P_{RF}$, power transferred to the plasma by ohmic heating, neutral beam injector and radio frequency respectively. \cite{SalarElahi2010} \cite{Fallis2013} The comparison of confinement is usually done before and after the appearance of the X-point radiator, or with a series of discharges in similar conditions but without X-point radiator.
The confinement degradation can be due to the direct cooling caused by the cold region (e.g. temperature poloidal gradients to drive power to that location \cite{Lipschultz1998} ) or to easier penetration of impurities in the core [\cite{Lipschultz2016} and references therein]. The degradation significantly affects the outer part of the core region with significant reduction of temperature, pressure and an increase of density. \cite{Kallenbach2015a}  It was found on some experiments that the inner core (ratio of poloidal magnetic flux over poloidal magnetic flux at the separatrix $ \rho _{pol}<0,8-0,5 $) is only marginally effected. [\cite{Reinke2013} and reference therein]It is therefore suggested that the gradients lost on the pedestal region could be recovered in a portion of plasma $0,8< \rho _{pol}<0,95$. \cite{Reimold2015}

The X-point radiator will also have the effect of radiating a substantial fraction ($75-90\%$ of the heating power achieved \cite{Bernert2017}) of the total $ \alpha $ particles / heating power out of the separatrix and could cause the transition to L-mode. This is anyway not a major risk for ITER and larger machines, because the power crossing the separatrix is expected to be significantly larger that the threshold requirement. It has been in fact considered to seed a higher-Z impurity in the plasma to enhance core radiation (radiative mantle) and lower the power needed to be exhausted through the separatrix to about the minimum for the L-H transition. Then a lower-Z impurity will be fed to radiate in the divertor. [\cite{Kallenbach2015a},\cite{Reinke2013} and reference therein]


There are two active areas of research to limit confinement losses.

One is in trying to achieve detachment but not allow the radiator moving all the way to the X-point and causing confinement degradation. The main difficulty in achieving this is that, for standard divertor geometries the operational window of any given control parameter to move the radiator from target to X-point is very narrow. Considerable effort has been put forward to find a predictive model. The thermal front model proposed in \cite{Lipschultz2016} is a 1D model that, balancing input / output power on the thermal front (edge of the radiator toward the core), tries to identify the operational window of a control parameter and its stability for given plasma / magnetic configuration. This model finds that for all the control variables analysed the operational window widens for increasing ratio of X-point over target magnetic field $B_x/B_t$ and extent ratio of the connection length between X-point and target over upstream and target $z_X/L$. It is also found that increasing connection length should lower the detachment upstream density threshold. Stability analysis indicates that for decreasing $z_X/L$ there is an increasing minimum Bx/Bt for a stable solution.\cite{Lipschultz2016}

This puts additional emphasis on research on different divertor designs. TCV in Switzerland and MAST-U in UK are the best suited for this type of investigations due to the flexibility of their divertor geometry. Recent data from TCV seems to prove that the sensitivity on control parameters decreases with flux expansion (larger operational window) but didn’t verify the threshold dependence on connection length. \cite{Theiler2017}

A second strategy is to live with the radiator located at the X-point (it’s further apart from the target, ELMs have to burn through more divertor volume before reaching the target) and try to understand and minimize the loss of confinement in the core. The reality of present conventional tokamaks is that the X-point radiator always appears if full detachment from the target is pursued, with different flavours depending on the seeded impurity as it can be seen in \autoref{fig:xprs}.

\begin{figure}
	\centering
	\includegraphics[width=\linewidth]{Chapters/chapter1/figs/xprs.png}
	\caption{Different radiation profiles with clear presence of X-point radiator for different impurities
	\cite{Wiesen2017}}
	\label{fig:xprs}
\end{figure}

The X-point radiator cannot be idealised as easily as the thermal front of detachment, because it is much more related to core and edge dynamics. Modeling its behaviour requires the use of codes that accounts for all atomic interactions, drifts, etc. like SOLPS-ITER, EDGE2D-EIRENE, SOLEDGE2D-EIRENE [\cite{Wiesen2017a} and references therein]. The presence of the radiator significantly affects temperature, pressure and density distribution, especially in the pedestal, therefore it is likely to have an effect on the current distribution and MHD activity.
In the last years a large effort has been put forward in the characterisation of the behaviour of the X-point radiator and its macroscopic effects on the core / edge. This has been done mostly in conventional geometry tokamaks, both with metal and carbon wall.


\section{ELMs and XPR}
ELM suppression/buffering
Another important feature of high levels of detachment of the XPR is the reduction in amplitude of ELMs. This can be attributed to the fact that the energy associated with it first heats up the radiator and then moves toward the target. If the neutral density between X-point and target, and in the radiator itself, is high enough it could be possible to avoid ELMs to “burn through” the detachment front and to reach the target altogether. \cite{Krasheninnikov2016}







\section{Goals and objectives of the thesis}

\subsection{XPR/radiation front location and confinement (maybe the dataset is not there)}
find limit where confinement is not compromised, correlation between confinement and XPR/radiation location
\subsection{poloidal divertor after XPR}
what it is?
\subsection{XPR/detachment and power balance}
radiator location vs radiated power
\subsection{radiation front location and analytic models}
DOD vs radiator location, radiation front range/stability vs prediction
\subsection{detachment and configuration (CD/SXD)}
radiated power vs configuration (against predictions)
\subsection{radiation location and other metric of detachment}
radiator location vs MWI/LP compared to expectations, ionisation/MAR/recombination region (Kevin work)
\subsection{XPR and ELMs}
ELMs burn through/impact on LP and IR vs detachment/radiation front location
\subsection{Cyd Cowley paper}
hysteresis in inner leg detachment
\subsection{ELMs baffled in linear machine}
Magnum work, it is possible to prevent ELMs to reach the target with target pressure
\subsection{atomic vs molecular effects during ELMs burn through in deep detachment in linear machine}



This is an example of how to reference \cite{VanEck2018}.

This is an example of how to place a figure in the text.




\chapter{MAST-U IRVB hardware activities}\label{chapter2}
This is an example of another chapter. Much like in chapter \ref{chapter1}.

\chapter{Magnum PSI activities}\label{chapter3}
\section{linear machine literature}
\section{utility of linear machine to tokamak detachment plasma physics}
\section{temperature/density dependent processes (atomic vs molecular)}
\subsection{ADAS}
\subsection{Yacora}
\section{Magnum capabilities}
\section{goals}
\section{experimental setup}
\subsection{fast camera}
\subsection{TS}
\subsection{IR camera}
\subsection{ADC}
\subsection{OES}
\subsubsection{HW improvement}
\subsubsection{data interpretation}
\subsection{sampling strategy}
\section{experiments results}
\subsection{fast camera}
\subsection{TS}
\subsection{IR camera: ELMs prevented from reaching the target}
\section{Bayesian technique}
\subsection{Hydrogen line emission fitting}
\subsection{power balance}
\subsubsection{ADC to input power}
\subsubsection{atomic/molecular/potential components}
\subsection{particle balance}
\subsubsection{input condition}
\subsubsection{atomic/molecular/potential components}
\subsection{priors}
\subsubsection{2D modelling: H2, H}
\subsubsection{AMJUEL:  H2+/ H2, H-/ H2}
\subsection{reaction rates}
\subsubsection{ADAS}
\subsubsection{Yacora}
\subsubsection{Aladdin (Janev)}
\section{Bayesian analysis results}
\subsection{radiated losses increase with pressure}
\subsection{relevance of molecular processes increase with pressure}
 \subsection{More important H2+ or H-?}




\section{MAST-U and the IRVB diagnostic}\label{MAST-U and the IRVB diagnostic}
\hl{to do}

\section{ELM-like pulses in Magnum-PSI}\label{ELM-like pulses in Magnum-PSI}
\hl{to do, this is copy of paper}
The effect of ELM-like pulses on a detached target in Magnum-PSI was studied with the help of various diagnostics. It was found that the ELM-like pulse energy can be effectively dissipated for high level of detachment before it reaches the target. The decreasing power reaching the target seems related to higher volumetric power losses in the volume between target and source. They seem initially fairly uniform becoming more and more uneven increasing the target chamber neutral pressure further, correlated with increased loss of plasma before reaching the location where it is measured. This behaviour can be divided in stages. In Stage 1 the plasma is attached before and during the ELM-like pulse and the energy losses in the volume are at a minimum and predominantly not radiative. In Stage 2 the target chamber neutral pressure is such that the plasma is detached from the target in steady state but can reattach during the ELM-like pulse. Radiative energy losses in the volume increase to dominate over other mechanisms, mainly driven by molecular assisted reactions. In Stage 3 the plasma is detached before and during the pulse and the losses in the volume are such that the plasma cannot reach the target, being effectively baffled.

Hydrogen Balmer line emission from optical emission spectroscopy was used to develop a Bayesian routine that incorporates organically the results from multiple diagnostics to return the power balance in the plasma column in the target chamber. Various approximations to extrapolate the results from a single location in the target chamber to its entirety were adopted. The routine incorporates priors from numerical simulations and collisional radiative codes so that atomic interactions can be distinguished from molecular ones. It was found that radiation from the plasma due to molecular assisted reactions is an important but not dominant energy loss mechanism. Mutual neutralisation of ${H}^-$ seems to dominate radiative losses, but it was established that this cannot be determined from OES alone with the present setup. Molecular assisted reactions significantly effect the local power balance via the exchange of potential energy, limiting what is available for ionisation. MAD is the most significant path via molecular precursors, but this could be due to an unconstrained paticle balance for $H_2$ and $H$. Molecular processes are mostly dominant in an intermediate temperature range around 3eV, between ionisation and EIR dominant regions.

These results indicate that for highly detached regimes in linear machines, and most likely also Tokamaks, molecular interactions are important and need to be accounted both in terms of potential energy exchange and induced hydrogenic line radiation, something that it’s not yet fully done in many codes used for Tokamak edge plasma studies.


% Anything after this line will be an appendix
\appendix

%\chapter{IRVB technique}\label{appendix1}
%I will explore here more technical details on the design and calibration explored as part of this project and necessary to extracting scientific information from the IRVB.

\section{Issues of the IRVB implementation in MU01}\label{Issues of the IRVB implementation in MU01}

It was realised when the data from super- experiment was analysed that the LOS that enter the super-X chamber have all a similar path through it, causing a large uncertainty on the source of the emissivity there. This can be shown by placing a phantom with known emissivity in the super-x chamber, determining the power on the foil and invert back to the emissivity. The result is shown in \autoref{fig:sxd_bad}.

\begin{figure}
     \centering
     \begin{subfigure}{0.45\textwidth}
         \centering
         \includegraphics[trim={0 0 30 0},clip,width=\textwidth]{Chapters/appendix1/figs/phantom_sdx.png}
         %\caption{SXD 45239}
         %\label{fig:table45239}
     \end{subfigure}
     \hfill
     \begin{subfigure}{0.45\textwidth}
         \centering
         \includegraphics[width=\textwidth]{Chapters/appendix1/figs/inversion_sdx.png}
         %\caption{CD 45401}
         %\label{fig:table45401}
     \end{subfigure}

    \caption{Phantom of radiation localised in the SXD and tomographically inverted emissivity, showing that the present setup is unable to reconstruct detailed emissivity maps of the radiation in the SXD.}
    \label{fig:sxd_bad}
\end{figure}


The inversion routine cannot locate the source of the emission, but the total radiated power in the chamber is still within $\approx 30\%$ of the input.

\subsection{MU02 geometry optimisation}\label{MU02 geometry optimisation}

After the results of the first MASTU campaign it was clear that with appropriate binning the SNR for relatively weak ohmic discharges is large enough (with binning of 14 temporal steps, 7 per digitizer, and 3x3 spatial peak signal on the foil $\approx 40W/m^2$ with uncertainty \hl{$\approx 7W/m^2$}) for high resolution emissivity maps. One negative result is that, because of the small angular difference between the different lines of sights (LOSs) entering the super-x chamber the resolution there is insufficient to reconstruct an emissivity map.
To define the optimal geometry for good SNR two test cases were found:
\begin{itemize}
    \item a ohmic SXD discharge with weak widespread radiation with the purpose of reconstructing the broad characteristics but mostly the total and partial sums of the radiated power. I decided to use 45239 at t=493ms
    \item a conventional high power H-mode discharge from a strong density scan and with x-point radiator, with strong signal, for the purpose of accurately reconstruct the radiation around the x-point. I decided to use 45401 at t=714ms
\end{itemize}

To perform the comparison the emissivity phantoms were calculated for each case with a binning of 14 temporal steps, 7 per camera digitizer, and 3x3 spatial ones with the present geometry. The negative emissivity was set to 0; to recreate a more realistic (field aligned) emissivity profile the emissivity was set to zero at $r>1m$ and $z>-1.5m$ for the 45239 phantom, while at $r>1m$ for 45401. The phantoms created are shown in \autoref{fig:phantoms}. The black dashed lines show the regions that will be used to compare the results.

\begin{figure}
     \centering
     \begin{subfigure}{0.45\textwidth}
         \centering
         \includegraphics[trim={70 20 0 20},clip,width=\textwidth]{Chapters/appendix1/figs/45239_2.png}
         \caption{SXD 45239}
         \label{fig:45239}
     \end{subfigure}
     \hfill
     \begin{subfigure}{0.45\textwidth}
         \centering
         \includegraphics[trim={70 20 0 20},clip,width=\textwidth]{Chapters/appendix1/figs/45401_2.png}
         \caption{CD 45401}
         \label{fig:45401_app}
     \end{subfigure}

    \caption{Phantoms used to compare the effect of different geometries on the tomographic inversion process. Left: shot 45239, 493ms, SXD discharge, emissivity=0 at $r>1m$ and $z>-1.5m$. Right:  shot 45401, 714ms, CD discharge, emissivity=0 at $r>1m$. The dashed lines represent the regions over which the results will be compared.}
    \label{fig:phantoms}
\end{figure}

The configurations investigated are all combinations of 45 and 60mm standoff and 4 and 6mm pinhole. 
For each configuration the geometry matrix is calculated using CHERAB to obtain the power absorbed by the foil. The temperature was then calculated with the same phantom for a number of time steps equal to 4 times the binning and the count noise added. The temperature profile is then binned and the power to the foil is calculated. The tomographic inversion is performed with the intermediate time step to find the new emissivity profile. The profiles are compared to the phantom in terms of local emissivity and integrated power. The results are shown in \autoref{fig:table_mu02}.

\begin{figure}
     \centering
     \begin{subfigure}{0.45\textwidth}
         \centering
         \includegraphics[width=\textwidth]{Chapters/appendix1/figs/table45239.png}
         \caption{SXD 45239}
         \label{fig:table45239}
     \end{subfigure}
     \hfill
     \begin{subfigure}{0.45\textwidth}
         \centering
         \includegraphics[width=\textwidth]{Chapters/appendix1/figs/table45401.png}
         \caption{CD 45401}
         \label{fig:table45401}
     \end{subfigure}

    \caption{Comparison of the integrated power $\%$ error for all geometric configurations for the SXD phantom from 45239 (a) and of the averaged power error std for all geometric configurations for the CD phantom from 45401 (b). CD region results are without bracket, SXD in with round brackets and x-point ones with square brackets}
    \label{fig:table_mu02}
\end{figure}

A larger pinhole allows a more precise accounting of the integrated power. Given the present SNR though, the improvement is minor. This is even more true considering that in MU02 the discharges are expected to be more energetic, with beam power even in L-mode. The power accounting in the SXD also improves with a longer stand-off.
The 60mm stand-off 4mm pinhole diameter configuration yields the better performance in terms of resolution around the x-point with strong signal. For high SNR additional LOSs count more than an even stronger signal.

Regarding the loss of spatial resolution in the SXD the stonger the signal the better. For larger pinhole the reconstructed radiation is more localised but still elongated along the LOS. Considering the design of IRVB aimed at a high resolution around the x-point and not the SXD, this is considered less important. It will be important to remember that the data inside the SXD is useful for integrated measurements rather than profiles.
With these consideration it was decided to adopt the 4mm diameter pinhole and 60mm long stand-off. It is the solution that causes the lower signal strength, but it is still deemed sufficient.

\section{IRVB calibration}\label{IRVBcalibration}


\subsection{counts oscillation suppression}
not sure if this is necessary


\chapter{Details on the Magnum-PSI study}\label{appendix2}


I will here expand in more detail on the models, approximations and assumptions that have been used in \autoref{chapter4}.

\section{Sampling strategy}\label{Sampling strategy}

As mentioned above, of all diagnostics only the fast camera has a time resolution high enough to resolve individual ELM-like pulses. The TS laser is fired at a fixed frequency of 10Hz from a dedicated timing system. The CB can be triggered at an arbitrary time compared to the 10Hz clock so that information on different parts of the pulse can be collected. The OES is triggered with the same 10Hz signal as TS and, due to the rolling shutter, acquires data with a time shift between rows pairs of $20\mu s$ for a total of $22ms$ required for exposure and acquisition of a single frame. For all experimental conditions it was decided to record TS/OES data from $0.5ms$ before the ELM-like pulse to $2.5ms$ after. A desired time resolution of $50\mu s$ determines $(22+3)/0.05=500$ ELM-like pulses, rounded to 600, are required. A phenomena often observed was that one capacitor failed to be triggered at the requested time, instead being trigger together with the next. This means that rather than 600 identical ELM-like pulses, one would be missing and the one after would be with a released energy twice the others. To reduce the effect of the missing data the required 600 pulses are split in two 300 pulses scans with the CB delayed of $100\mu s$ at the time. The two scans are shifted of $50\mu s$ to obtain the desired $50\mu s$ resolution for all the OES rows. With this strategy the maximum time separation between two ELM-like pulses with good data is $100\mu s$ rather than 150$\mu$s. The missing data is obtained by interpolating between the good data. In \autoref{fig:sampling1} is represented the sampling strategy adopted. At the top is indicated how the camera exposure and readout are shifted in time due to the rolling shutter. 

\begin{figure*}
	\centering
	\includegraphics[width=\linewidth,trim={0 0 0 0},clip]{Chapters/chapter3/figs/sampling_strategy_2.png}
	\caption{Sampling strategy for TS and OES. At the top is indicated the progressive reading of the camera rows, with the integration time of TS is indicated. Below are each of the ELM-like. pulses. It is shown how the trigger of the capacitor bank is shifted in time and all the pulses are split in two scans. Dashed lines indicate an interruption of the space or time scale.}
	\label{fig:sampling1}
\end{figure*}

The rows are managed two at the time and during the readout of the rows $i-1/i$ the rows $i+1/i+2$ are exposed. Each OES LOS is composed of $\sim 24$ rows, so the time shift within would be $240\mu s$, much larger than the desired $50 \mu s$. Just below is indicated the TS integration time. OES and TS are synchronised such that the trigger to start data acquisition is sent to the two diagnostics simultaneously.
The time difference between CB and OES/TS trigger is initially such that OES and TS reading correspond to the end of the ELM-like pulse. From this the CB trigger is progressively delayed by $100\mu s$ so that TS and OES can acquire data about progressively earlier stages of the pulse.
For TS this sampling procedure returns directly $T_e$ and $n_e$ while for OES additional steps are required to: reconstruct frames with row data corresponding to the same time slice, bin the LOS, obtain the emission line brightness, calculate radial emissivity from the line integrates brightness. Further details are given in \autoref{OES data interpretation}. The final result is a map of the line emissivity in both space (radius) and time. The plasma is assumed poloidally symmetric with a radial coordinate of interval 1.06mm dictated by the OES resolution.

\section{OES data interpretation}\label{OES data interpretation}

\begin{figure}[!ht]
	\centering
	\includegraphics[width=0.7\linewidth,trim={440 50 600 150},clip]{Chapters/chapter3/figs/sample_oes.png}
	\caption{Example of a raw image from the OES. The readout starts from the bottom so higher rows represent later times. In the figure are indicated rows/times corresponding to different stages of the ELM-like pulse}
	\label{fig:sampling2}
\end{figure}

In this section it will be detailed how the OES measurements are processes to obtain the local radially and temporally resolved emissivity used to estimate the relevance of molecular processes.
The camera that was selected for the purpose of collecting time resolved OES data was a Photometrics Prime95B 25mm RM16C, because of the relatively high signal to noise in low light conditions and large size of the sensor. In \autoref{fig:sampling2} is shown the typical picture collected during an experiment when on top of a steady state plasma an ELM-like pulse is fired. The rows are read sequentially from the bottom, with a time shift equal to the integration time, minimum $20\mu s$. This means that for the particular example shown the rows indicated as \emph{Before} represent times before the effect of the ELM-like pulse propagated to the OES location. \emph{During} represents the pulse and \emph{After} is for times after the ELM-like pulse, characterised by homogeneous line emission from the hot gas filling the target chamber. From \autoref{fig:sampling2} it is also possible to distinguish part of the 40 line of sight that are available.

The first effect that is compensated is the sensitivity of the camera at low signal levels. The counts/light intensity correlation of the pixels is mostly linear, but deviates significantly below 6 counts and negative counts are returned at very low signal. A routine was developed to compensate for this thanks to dedicated measurements to find the correlation between light intensity and counts.

In order to decouple spatial and temporal information a scan is operated such that the ELM-like pulse is shifted in time respect to the start of the camera image record and TS measurement. More details on the sampling strategy in \autoref{Sampling strategy}. The presence of ELM-like pulses effected by capacitor bank misfires (see \autoref{Sampling strategy}) is found analysing the plasma source power and the data corresponding to those pulses is excluded. To separate the time and row dependency, for every row, column and time of interest the data in a range of $100\mu s$ and 8 rows is fit with a second degree polynomial in time and one in row. To avoid over smoothing the image a smaller weight is assigned for increasing times and row difference from the one that is being examined. In \autoref{fig:sampling3} is shown the time/row decoupled image. The output time step has a $50\mu s$ resolution to match TS data.

\begin{figure}[!ht]
	\centering
	\includegraphics[width=0.7\linewidth,trim={100 30 290 200},clip]{Chapters/chapter3/figs/sample_oes2.png}
	\caption{Example of decomposed time frame showing the symmetry of the image to the vertical pixel $\sim$600, likely representing the location of the plasma column axis.}
	\label{fig:sampling3}
\end{figure}

The counts are summed among the rows composing each LOS and the line intensity is calculated by integrating above the background level. Brightness is then converted to emissivity via Abel inversion. The line emission is supposed poloidally symmetric and the plasma optically thin. In order to avoid unrealistic discontinuities given by noise, the superimposition of 3 Gaussian is fitted to the brightness profile as done by Barrois.\cite{Science2017} Each Gaussian can then be Abel inverted analytically and summed to obtain the total emissivity. In this process the uncertainties are propagated to be used in subsequent steps in the analysis. An example of the inversion process is shown in \autoref{fig:sampling4}. Given the signal to noise ratio and the available lines it is decided to use Balmer lines $p=4-8 \rightarrow 2$.

\begin{figure}[!ht]
     \centering
     \begin{subfigure}{0.8\linewidth}
         \centering
         \includegraphics[width=0.7\textwidth,trim={20 170 550 300},clip]{Chapters/chapter3/figs/line_integrted_profile4.png}
         % \vspace*{-8mm}
         \caption{OES line integrated brightness}
         \label{fig:sampling4a}
         %{\color{white}\caption{\phantom{wew}}\label{fig:TSa}}
     \end{subfigure}
     % \hfill
     \begin{subfigure}{0.14\linewidth}
         % \centering
         \vspace*{-10mm}
         \hspace*{-20mm}
         \includegraphics[width=0.7\textwidth,trim={2200 0 0 40},clip]{Chapters/chapter3/figs/line_integrted_profile4.png}
         % \vspace*{-8mm}
         % \caption{Line integrated brightness}
         % \label{fig:sampling4a}
         %{\color{white}\caption{\phantom{wew}}\label{fig:TSa}}
     \end{subfigure}
     % \hfill
     \begin{subfigure}{0.8\linewidth}
         \centering
         \includegraphics[width=0.7\textwidth,trim={20 220 650 400},clip]{Chapters/chapter3/figs/line4SNR.jpg}
         % \vspace*{-8mm}
         \caption{OES line integrated brightness SNR}
         \label{fig:sampling4b}
         %{\color{white}\caption{\phantom{wew}}\label{fig:TSb}}
     \end{subfigure}
     % \hfill
     \begin{subfigure}{0.095\linewidth}
         % \centering
         \vspace*{-10mm}
         \hspace*{-20mm}
         \includegraphics[width=0.7\textwidth,trim={3000 0 125 40},clip]{Chapters/chapter3/figs/line4SNR.jpg}
         % \vspace*{-8mm}
         % \caption{Line integrated brightness}
         % \label{fig:sampling4a}
         %{\color{white}\caption{\phantom{wew}}\label{fig:TSa}}
     \end{subfigure}
        \caption{(\subref{fig:sampling4a}): example of a fit of the OES line integrated brightness and relative SNR with 3 Gaussians for the $n=4\rightarrow 2$ line. In magenta the profile of the fitted Gausinass, in thin black the found location of the plasma column centre. (\subref{fig:sampling4b}): SNR of the line integrated brightness.}
        \label{fig:sampling4}
\end{figure}


\section{Target temperature profile interpretation}\label{Target temperature profile interpretation}
In this section the mathematical models used in the interpretation of the target temperature data will be examined in more detail.

The heat delivered by the plasma to the target during an ELM-like pulse is shaped in time and space. The power density is spatially peaked at the centre of the plasma beam due to the peak in plasma density and temperature. It temporally follows the power evolution dictated by the discharge of the capacitor bank as measured by the plasma source (see \autoref{fig:TS1} for an example). The full spatial and temporal power density distribution of the target heat source is difficult to obtain from the surface temperature for all experimental conditions but analytical solutions that can approximate the peak temperature are available. As mentioned in \autoref{IR camera} the ones considered in this work are:

\begin{itemize}
\item[Model1] a temporally square heat wave delivered uniformly on a semi infinite plane: the surface temperature increase in the heating and cooling phases are respectively\cite{Behrisch1980}
\begin{equation}
\label{eq:square1}
\begin{aligned}
{\Delta T(t)}_r &= F_0 \frac{2}{ \sqrt{\pi \rho c_p k }} \lbrace { \sqrt{t-t_0} } \rbrace = \frac{E_0}{\tau} \frac{2}{ \sqrt{\pi \rho c_p k }} \lbrace { \sqrt{t-t_0} } \rbrace \\ {\Delta T(t)}_c &= F_0 \frac{2}{ \sqrt{\pi \rho c_p k} } \lbrace { \sqrt{t-t_0} - \sqrt{t-t_0 - {\tau}} } \rbrace
\end{aligned}
\end{equation}
with $\tau$ the duration of the heat pulse and $t_0$ its start, $F_0$ the peak power density, $E_0$ the energy density, $\rho$ the density, $c_p$ the specific heat capacity and $k$ the thermal conductivity
\item[Model2] a temporally square heat wave delivered by a spatially Gaussian beam: the surface temperature evolution in the heating and cooling phases are respectively\cite{Bauerle2011}
\begin{equation}
\label{eq:square2}
\begin{aligned}
{\Delta T(t)}_r &= \frac{2}{\pi} {\Theta}_c \tan^{ -1}(2 \sqrt{ \tilde{t}} ) 
\\ 
{\Delta T(t)}_c &= \frac{2}{\pi} {\Theta}_c \tan^{ -1} \left( \frac{2 \sqrt{ \tilde{t}} - 2 \sqrt{ \tilde{t} - \tilde{\tau} }}{ 1+4 \sqrt{ \tilde{t}} \sqrt{ \tilde{t} - \tilde{\tau} }} \right)
\end{aligned}
\end{equation}
with
\begin{equation}
\label{eq:square3}
\begin{aligned}
{\Theta}_c = \frac{\sqrt{\pi}}{2} \frac{F_0 {\omega}_0}{k} , \tilde{t} = \frac{(t-t_0) D}{{\omega}_0} , \tilde{\tau} = \frac{\tau D}{{\omega}_0} , D = \frac{k}{ \rho c_p }
\end{aligned}
\end{equation}
and $\omega_0$ the $1/e$ size of the spatial heat distribution on the target
\item[Model3] a temporally triangular heat wave delivered by a spatially Gaussian beam: the surface temperature evolution in the heat rise, heat decreasing and cooling phases are respectively
% \begin{equation}
% \label{eq:triang}
% \begin{aligned}
% {\Delta T(t)}_r = \frac{2}{\pi} {\Theta}_c \frac{1}{\tilde{\tau_r}}   \left\{ {\tilde{t}} \tan^{ -1}(2 \sqrt{ \tilde{t}} )  \frac{-1}{4} [ 2 \sqrt{ \tilde{t}} - \tan^{ -1}(2 \sqrt{ \tilde{t}} ) ] \right\}
% \\ 
% {\Delta T(t)}_d = {\Delta T(t)}_r - \frac{2}{\pi} {\Theta}_c \left( \frac{1}{\tilde{\tau_r}} +\frac{1}{\tilde{\tau_d}}  \right) \left\{ {(\tilde{t} - \tilde{\tau_r})} \tan^{ -1} (2 \sqrt{ \tilde{t} - \tilde{\tau_r}} ) + \right. \\ \left. - \frac{1}{4 } [ 2 \sqrt{ \tilde{t} - \tilde{\tau_r}} - \tan^{ -1}(2 \sqrt{ \tilde{t} - \tilde{\tau_r}} ) ] \right\} 
% \\ 
% {\Delta T(t)}_c = {\Delta T(t)}_d + \frac{2}{\pi} {\Theta}_c \frac{1}{\tilde{\tau_d}} \left\{ {(\tilde{t} - \tilde{\tau_r} - \tilde{\tau_d})} \tan^{ -1}(2 \sqrt{ \tilde{t} - \tilde{\tau_r} - \tilde{\tau_d}} )+ \right. \\ \left. - \frac{1}{4 } [ 2 \sqrt{ \tilde{t} - \tilde{\tau_r} - \tilde{\tau_d}} - \tan^{ -1}(2 \sqrt{ \tilde{t} - \tilde{\tau_r} - \tilde{\tau_d}} ) ] \right\} 
% \end{aligned}
% \end{equation}


\begin{equation}
\label{eq:triang}
\begin{aligned}
{\Delta T(t)}_r &= \frac{2}{\pi} {\Theta}_c \frac{1}{\tilde{\tau_r}}   \left\{ {\tilde{t}} \tan^{ -1}(2 \sqrt{ \tilde{t}} )  \frac{-1}{4} [ 2 \sqrt{ \tilde{t}} - \tan^{ -1}(2 \sqrt{ \tilde{t}} ) ] \right\}
\\ 
{\Delta T(t)}_d &= {\Delta T(t)}_r - \frac{2}{\pi} {\Theta}_c \left( \frac{1}{\tilde{\tau_r}} +\frac{1}{\tilde{\tau_d}}  \right) \left\{ {(\tilde{t} - \tilde{\tau_r})} \tan^{ -1} (2 \sqrt{ \tilde{t} - \tilde{\tau_r}} )  \right. 
\\
& \phantom{+} \left. - \frac{1}{4 } [ 2 \sqrt{ \tilde{t} - \tilde{\tau_r}} - \tan^{ -1}(2 \sqrt{ \tilde{t} - \tilde{\tau_r}} ) ] \right\} 
\\ 
{\Delta T(t)}_c &= {\Delta T(t)}_d + \frac{2}{\pi} {\Theta}_c \frac{1}{\tilde{\tau_d}} \left\{ {(\tilde{t} - \tilde{\tau_r} - \tilde{\tau_d})} \tan^{ -1}(2 \sqrt{ \tilde{t} - \tilde{\tau_r} - \tilde{\tau_d}} ) \right. 
\\
& \phantom{+} \left. - \frac{1}{4 } [ 2 \sqrt{ \tilde{t} - \tilde{\tau_r} - \tilde{\tau_d}} - \tan^{ -1}(2 \sqrt{ \tilde{t} - \tilde{\tau_r} - \tilde{\tau_d}} ) ] \right\}
\end{aligned}
\end{equation}
with $\tilde{\tau_r}$ the time to rise the power density from $0$ to $F_0$ and $\tilde{\tau_d}$ the time to decrease to $0$ again.

\end{itemize}
These have been obtained assuming homogeneous and constant thermal properties and a semi infinite target. With these assumptions the heat equation is linear and the superposition principle could be employed.

An important observation from these analytical solutions is that in all cases in the limit $t>>\tau$ the surface temperature cooling is proportional to $t^{-3/2}$. For pulses with a significant energy delivered, the target surface is still cooling when the next ELM-like pulse comes, so the temperature has to be corrected. The temperature before the pulse is fit with
\begin{equation}
\label{eq:squareSS}
\begin{aligned}
T=\frac{a}{ (t+ \Delta t)^{3/2}} + T_{0}
\end{aligned}
\end{equation}
with $\Delta t$ the time between consecutive ELM-like pulses. The time dependent component is then subtracted.


\begin{figure}[!ht]
	\centering
	\includegraphics[width=0.7\linewidth,trim={750 550 10 115},clip]{Chapters/chapter3/figs/file_index_394_IR_trace_Capture049_43_old.png}
	\caption{Measured and fitted peak target temperature for a discharge in Stage 1 (ID 5 in \autoref{tab:table1}). In dashed green the fit of the temperature profile using the triangular Gaussian pulse model for t>1.5ms and in solid green the whole profile. The black dashed line indicates the cooling from the previous pulse and the thin solid black one is the peak target temperature not corrected for this.}
	\label{fig:IR4}
\end{figure}

\begin{figure}[!ht]
	\centering
	\includegraphics[width=0.7\linewidth,trim={5 5 55 65},clip]{Chapters/chapter3/figs/file_index_393_IR_trace_Capture048_46.png}
	\caption{Peak temperature of the target for low neutral pressure (ID 5 in \autoref{tab:table1}). To estimate the effect of the ELM-like pulse only the second half of the pulses is used, when the target is close to a steady state.}
	\label{fig:IR5}
\end{figure}

The result can be seen in the black dashed line in \autoref{fig:IR4} and in the increase of the peak temperature from the thin solid black line to the thick one. This correction is significant only for low neutral pressure conditions. In this cases it has to be taken also into account that it takes time to reach a thermal equilibrium after the ELM-like pulse train is started. To minimize this slow variation only the second 150 of the ELM-like pulses within a 300 strong scan is used to construct the average profile as it can be seen from \autoref{fig:IR5}.

\begin{figure}[!ht]
    % \hspace*{+10mm}
     \centering
     \begin{subfigure}{0.7\linewidth}
         \centering
        \hspace*{-5mm}
         %\includegraphics[width=\textwidth,trim={750 10 0 680},clip]{Chapters/chapter3/figs/file_index_394_IR_trace_Capture049_43.png}
         \includegraphics[width=0.8\textwidth,trim={510 5 5 415},clip]{Chapters/chapter3/figs/file_index_394_IR_trace_Capture049_43.png}
         % \vspace*{-30mm}
         \caption{\phantom{ww}}
         \label{fig:IR6a}
         %{\color{white}\caption{\phantom{wew}}\label{fig:TSa}}
     \end{subfigure}
     \hfill
    % \vspace*{+22mm}
    \begin{subfigure}{0.7\linewidth}
         \centering
         %\includegraphics[width=\textwidth,trim={5 10 770 680},clip]{Chapters/chapter3/figs/file_index_394_IR_trace_Capture049_43.png}
         \includegraphics[width=0.8\textwidth,trim={5 5 515 415},clip]{Chapters/chapter3/figs/file_index_394_IR_trace_Capture049_43.png}
         % \vspace*{-30mm}
         \caption{\phantom{wew}}
         \label{fig:IR6b}
         %{\color{white}\caption{\phantom{wew}}\label{fig:TSb}}
     \end{subfigure}
        % \vspace*{+20mm}
        \caption{Comparison of the peak power density profiles (\subref{fig:IR6b}) result of fitting the peak temperature curve (\subref{fig:IR6a}) the with different analytical solutions for the lowest available neutral pressure conditions (ID 5 in \autoref{tab:table1}). All find adequate values for the energy density even with radically different peak durations. Dashed lines are obtained fitting from 1.5 to 30ms while solid ones from -20 to 30ms. In magenta 1.5ms after the temperature peak.}
        \label{fig:IR6}
\end{figure}

Considering all this a fit of the ELM-like pulse with the lowest neutral pressure can be used to compare the performance of the different models. As shown in \autoref{fig:IR6} all models fit well the slowly decreasing peak temperature slope and return a power density within 10\% of each other. Model3 fits best also the peak when the entire time axis is used.

\begin{figure}[!ht]
    % \hspace*{+10mm}
     \centering
     \begin{subfigure}{0.7\linewidth}
         \centering
        \hspace*{-5mm}
         \includegraphics[width=0.8\textwidth,trim={510 5 5 330},clip]{Chapters/chapter3/figs/Figure_3.png}
         % \vspace*{-30mm}
         \caption{\phantom{wew}}
         \label{fig:IR7a}
         %{\color{white}\caption{\phantom{wew}}\label{fig:TSa}}
     \end{subfigure}
     \hfill
    % \vspace*{+22mm}
     \begin{subfigure}{0.7\linewidth}
         \centering
         \includegraphics[width=0.8\textwidth,trim={5 5 515 330},clip]{Chapters/chapter3/figs/Figure_3.png}
         % \vspace*{-30mm}
         \caption{\phantom{wew}}
         \label{fig:IR7b}
         %{\color{white}\caption{\phantom{wew}}\label{fig:TSb}}
     \end{subfigure}
        % \vspace*{+20mm}
        \caption{Peak target temperature evolution (\subref{fig:IR7a}) calculated with the MSC.Marc/Mentat® code from a known heat flux profile made to be similar to the one caused by the CB in Magnum-PSI (\subref{fig:IR7b}) (solid black lines). That is fit with different analytical solutions and the inferred heat flux compared to the input. Dashed lines are obtained fitting from 1.5 to 30ms while solid ones the entire time axis. In magenta 1.5ms after the temperature peak.}
        \label{fig:IR7}
\end{figure}

The same comparison can be done with a simulated temperature rise obtained from a known input power profile. The MSC.Marc/Mentat® non linear FEM suite was used to reproduce a heat pulse similar to the one generated by the CB in terms of temporal and spatial variation and with temperature dependent material properties. The spatial distribution of the heat pulse was set as a Gaussian of radial extent consistent with what measured with TS ($1/e^2 \sim 1$cm) while the temporal variation was reproduced by fitting two Gaussians to the power profile from the plasma source. The comparison of the fits using the 3 models is shown in \autoref{fig:IR7}.

Model3 is capable of better reproducing the full temperature profile returning, for the full fit, with a power density shape very close to the input one. Even if the pulse duration when fitting only after 1.5ms is quite different from the input, the energy density is within 10\% of the input one for all cases and, as mentioned before, the uncertainty on the energy delivered to the target is $\sim$20\%, good enough for the purpose of this work.

\begin{figure}[!ht]
    \captionsetup{labelfont={color=black}}
    % \hspace*{+7mm}
     \centering
     \begin{subfigure}{0.6\linewidth}
         \centering
         % \vspace*{-0mm}
         \includegraphics[width=\textwidth,trim={50 40 10 60},clip]{Chapters/chapter3/figs/file_index_393_IR_trace_Capture048_32b.png}
        \vspace*{-5mm}
        {\color{white}\caption{\phantom{wewew}}\label{fig:IR8a}}
     \end{subfigure}
     \hfill
    % \vspace*{+22mm}
     \begin{subfigure}{0.6\linewidth}
         \centering
         % \vspace*{-0mm}
         \includegraphics[width=\textwidth,trim={50 40 10 66},clip]{Chapters/chapter3/figs/file_index_267_IR_trace_Capture003_restrict_32b (copy).png}
        \vspace*{-5mm}
         {\color{white}\caption{\phantom{}}\label{fig:IR8b}}
     \end{subfigure}
        % \vspace*{+20mm}
        \caption{Peak temperature increase distribution over the steady state on the target. (\subref{fig:IR8a}) low neutral pressure conditions showing a clear confined peak (ID 5 in \autoref{tab:table1}). (\subref{fig:IR8b}) high neutral pressure case showing the temperature increase is not localized, possibly consistent with reflections of volumetric radiation or radiation of other origin (ID 10 in \autoref{tab:table1}). The black dashed line is the circular side of the cylindrical target. The arrow indicates where the source is outside the camera view.}
        \label{fig:IR8}
\end{figure}

To reinforce the argument on the origin of the prompt emission for high neutral pressure cases it is shown in \autoref{fig:IR8} the temperature profile at its peak for a low and high neutral pressure case. The low neutral pressure case (\subref{fig:IR8a}) shows a well defined peak with a radially symmetric decreasing profile. The high neutral pressure case (\subref{fig:IR8b}) is very different, with a much wider peak and a high temperature shadow from the peak to the edge of the target (marked in black). Note also the higher temperature outside the target itself. Considering the camera looks at the target at an angle and that the source is at the left of the field of view (see \autoref{fig:layout}), it is possible that the prompt peak could be due to radiation from the plasma itself reflected by the target. A definitive answer on what the origin of this radiation is cannot be given at this stage, but this observation further strengthen the case for only fitting the slowly decreasing temperature profile after the pulse.

\section{Details on the Bayesian calculations}\label{Details on the Bayesian calculations}

It will be detailed here how the expected properties of a plasma given a set of priors are calculated, compared with the measurements and used in the particle and power balance.

\subsection{Priors from B2.5 Eunomia}\label{Priors from B2.5 Eunomia}

In order to define the initial parameter space and the prior it is necessary to define the range and probability associated with all the axis of the parameter space.

For $T_e$ and $n_e$ the TS values are used and the range is defined as 6 times the uncertainty. The probability is calculated as a linear normal distribution with the uncertainty corresponding to 1 sigma.

The range and probabilities for $n_{H_2}/n_e$ are obtained from B2.5-Eunomia simulations for a steady state neutral pressure scan with 2 plasma source settings, ranging from attached to detached via increasing neutral hydrogen target chamber neutral pressure, carried out by Chandra.\cite{Chandra2021,Chandra2022} The simulations consider the whole plasma column source to target, but only data inside the target chamber and within 2cm of the axis are considered here (marked with $x$, versus other regions marked by a point in \autoref{fig:priors1}, \ref{fig:priors2}, \ref{fig:priors3}, \ref{fig:priors4}, \ref{fig:priors4b}, \ref{fig:priors5}).

\begin{figure}[!ht]
	\centering
	\includegraphics[width=0.7\linewidth,trim={0 0 30 45},clip]{Chapters/chapter3/figs/nH2_ne3.png}
	\caption{Correlation between molecular hydrogen and plasma density with temperature from B2.5-Eunomia modelling. The colored dashed lines are obtained with a linear log log fit for the single cases. The solid black line is obtained averaging the fitting parameters obtained.}
	\label{fig:priors1}
\end{figure}


As demonstrated by Den Harder\cite{DenHarder2015} the density decrease of $H_2$ in the plasma is mainly driven by rarefaction due to the high temperature of the plasma itself. For this reason a quite strong correlation between molecular hydrogen and plasma density ratio  and plasma temperature exists, shown in \autoref{fig:priors1}. The most likely ratio is obtained by fitting each simulation's results with a linear log log function and then averaging the fit parameters as shown in \autoref{fig:priors1}. The probability is defined as a normal distribution where 2 sigma corresponds to the black dashed lines, 100 and 1/100 times the fit value. The range is a significantly larger window around the dashed lines, to account for the large uncertainty coming from the fact that B2.5-Eunomia only simulates steady state conditions while the ELM-like burn through is a very dynamic one.

\begin{figure}[!ht]
	\centering
	\includegraphics[width=0.7\linewidth,trim={0 0 30 45},clip]{Chapters/chapter3/figs/nH_ne3.png}
	\caption{Correlation between atomic hydrogen and plasma density with temperature from B2.5-Eunomia modeling. The colored dashed lines are obtained with a linear log log fit for the single cases. The solid black line is obtained averaging the fitting parameters obtained.}
	\label{fig:priors2}
\end{figure}

The simulations are used to provide also range and probability for $n_H/n_e$. Atomic hydrogen is generated from recombination and from $H_2$ interaction with plasma and various molecules, so its density is only weakly correlated with plasma temperature and density, as shown in \autoref{fig:priors2}. 
The probability was calculated with a linear normal distribution with nominal value from the fit (calculated in the same fashion as for $n_{H_2}/n_e$) and 2 sigma arbitrarily assigned as per the dashed line in \autoref{fig:priors2}.
% no, this was reversed Given the large range and spread in $n_H/n_e$, the use of the probability for $n_H/n_e$ was later dropped and instead a linear uniform probability for all values was used.

\begin{figure}[!ht]
	\centering
	\includegraphics[width=0.7\linewidth,trim={0 0 30 45},clip]{Chapters/chapter3/figs/TH_Te3.png}
	\caption{Correlation between atomic hydrogen and electron temperature from B2.5-Eunomia modeling}
	\label{fig:priors3}
\end{figure}

\begin{figure}[!ht]
	\centering
	\includegraphics[width=0.7\linewidth,trim={0 0 30 45},clip]{Chapters/chapter3/figs/TH2_Te3.png}
	\caption{Correlation between $H_2$ and electron temperature from B2.5-Eunomia modelling. The colored dashed lines are obtained with a linear log log fit for the single cases. The solid black line is obtained averaging the fitting parameters obtained. This is assumed to be the same as ${H_2}^+$ temperature, while to obtain $H^-$ temperature 2.2eV are added to account for the $H_2$ dissociation energy.}
	\label{fig:priors4}
\end{figure}

\begin{figure}[!ht]
	\centering
	\includegraphics[width=0.7\linewidth,trim={0 0 30 45},clip]{Chapters/chapter3/figs/TH2_Te_ne3.png}
	\caption{Residuals from fitting $T_{H_2}$ with the scaling from \autoref{fig:priors4} and their weak dependence on the plasma density. The linear log log scaling in black is obtained by fitting all points at once.}
	\label{fig:priors4b}
\end{figure}

\begin{figure}[!ht]
	\centering
	\includegraphics[width=0.7\linewidth,trim={0 0 30 45},clip]{Chapters/chapter3/figs/THp_Te3.png}
	\caption{Correlation between $H^+$ and electron temperature from B2.5-Eunomia modelling. The dashed line indicates $T_{H^+}=T_e$.}
	\label{fig:priors5}
\end{figure}

Other quantities that are part of the plasma state and had to be determined to calculate reaction rates and other coefficients are the temperatures of all species. To reduce the number of variables in the Bayesian algorithm their uncertainty is in this work not considered and only the nominal values are used. The correlations are shown in \autoref{fig:priors3}, \ref{fig:priors4}, \ref{fig:priors5} for $H$, $H_2$ and $H^+$ temperature respectively where the black dashed lines indicates the values used. For $T_H$ and $T_{H_2}$ the fit is obtained in the same fashion as $n_H/n_e$ while for $T_{H^+}$ it is assumed $T_{H^+}=T_e=T_{plasma}$. For $T_{H_2}$ in particular a weak dependence on the plasma density is present, whose estimate is shown in \autoref{fig:priors4b} and can be due to an increase of the collisionality for higher density and a better coupling with the neutrals, resulting in a correction factor to apply to the dependency on $T_e$ alone. Given ${H_2}^+$ is mostly originated from $H_2$ it is considered $T_{H_2}=T_{{H_2}^+}$. This is valid for $H^-$ too, but because it can get some of the $H_2$ binding energy (2.2eV per atom) 2.2eV are added to $T_{H_2}$ to estimate $T_{H^-}$.\cite{Verhaegh2020}


\subsection{Priors from AMJUEL}\label{Priors from AMJUEL}
Ionised hydrogen molecules are generated mainly from $H_2$ so their density prior is calculated with AMJUEL\cite{Reiter2017}, a library that, among others, contains $n_{H^-}/n_{H_2}$ and $n_{{H_2}^+}/n_{H_2}$ density ratios in an equilibrium plasma for given plasma temperature and density (Section 12.58, 12.59, 11.11, 11.12). The conditions of an ELM-like pulse can potentially deviate significantly from equilibrium, so a wide range around equilibrium is considered as prior and a linear uniform distribution as likelihood.


\subsection{Priors range optimization}\label{Priors range optimization}
In order to optimize the $n_{H}/n_e$ range to only useful values a combination of the information from \autoref{fig:priors1} and the OES measurement is used. For each $T_e$ / $n_e$ combination it is calculated what is the emission from EIR via the ADAS PEC coefficients and subtracted from the OES measurement. It is then calculated what is the $n_{H}/n_e$ required to recreate via EIE each residual line emission plus its uncertainty. The largest $n_{H}/n_e$, limited by a predefined multiplier times the value from the B2.5-Eunomia fit, will be the highest value considered for that particular $T_e$ / $n_e$ combination. The lower limit will be taken as a small fraction of the maximum value, again limited by a predefined multiplier times the value from the B2.5-Eunomia fit. In this way parts of the range of $n_{H}/n_e$ that would cause an excessive line emission are automatically excluded and the prior range is assigned efficiently.

The same process is applied to the $n_{H_2}/n_e$ prior. For each $T_e$ / $n_e$ / $n_{H}/n_e$ the total emission from EIR and EIE is subtracted from the OES measurement and the $n_{H_2}/n_e$ required to match the residual is calculated with the Yacora coefficients for the $H_2$ dissociation reaction. For the $n_{{H_2}^+}/n_{H_2}$ prior the emission from EIR, EIE, $H_2$ dissociation is considered. Consequently for the $n_{{H}^-}/n_{H_2}$ prior also the emission from ${H_2}^+$ is considered.

\subsection{Emissivity}\label{Emissivity}

The emissivity is calculated for known precursors densities via the ADAS PECs and Yacora population coefficients.\cite{Verhaegh2020}
The Photon Emission Coefficients (PEC, photons $m^3/s$) coefficients are defined as the number of photons generated per second per unit of the precursors density. The number of photons for the transition $p \rightarrow q$ is equal to the product of density of the excited state $p$ and the Einstein coefficient $A_{pq}$ so the emission generated by atomic excitation and recombination can be expressed as per \autoref{eq:emiss1} and \ref{eq:emiss2}, where it is also highlighted what is intended as population coefficient ($PC_i$).

\begin{equation}
\label{eq:emiss1}
\begin{aligned}
\epsilon^{exc}_{pq} = PEC^{exc}_{pq}(T_e,n_e) n_e n_{H} = A_{pq} \underbrace{ \frac{n_{H(p)}}{n_e n_{H}}}_{PC_{exc}} n_e n_{H} 
\end{aligned}
\end{equation}

\begin{equation}
\label{eq:emiss2}
\begin{aligned}
\epsilon^{rec}_{pq} = PEC^{rec}_{pq}(T_e,n_e) n_e n_{H^+{}} = A_{pq} \underbrace{\frac{n_{H(p)}}{n_e n_{H^+{}}}}_{PC_{rec}} n_e n_{H^+{}}
\end{aligned}
\end{equation}

The line emission due to molecular reactions is similarly calculated via the Yacora population coefficients as per \autoref{eq:emiss3}, \ref{eq:emiss4}, \ref{eq:emiss5} and \ref{eq:emiss6}. It is also shown which reaction was considered in building the coefficients, and the variables necessary to calculate the coefficients. As mentioned $T_H$, $T_{H_2}$ are determined from the B2.5-Eunomia simulation while $T_{H_2} \approx T_{{H_2}^+} \approx T_{H^-}-2.2eV$.

\begin{equation}
\label{eq:emiss3}
\begin{aligned}
\epsilon^{{H_2}^+{}}_{pq} =& A_{pq} PC_{{H_2}^+{}}(T_e,n_e) n_e n_{{H_2}^+{}} \\
reactions:\ &{H_2}^+{} + e^-{} \rightarrow H(p) + H^+{} + e^-{} \\ 
\ &{H_2}^+{} + e^-{} \rightarrow H(p) + H(0)
\end{aligned}
\end{equation}

\begin{equation}
\label{eq:emiss4}
\begin{aligned}
\epsilon^{{H_2}}_{pq} =& A_{pq} PC_{{H_2}}(T_e,n_e) n_e n_{{H_2}} \\
reaction:\ &{H_2}^+{} + e^-{} \rightarrow H(p) + H(1) + e^-{}
\end{aligned}
\end{equation}

\begin{equation}
\label{eq:emiss5}
\begin{aligned}
\epsilon^{{H}^-{}+{H_2}^+{}}_{pq} =& A_{pq} PC_{{H}^-{}+{H_2}^+{}}(T_e,T_{{H_2}^+{}},T_{{H}^-{}},n_e) n_{{H_2}^+{}} n_{{H}^-{}} \\
reaction:\ &{H}^-{}+{H_2}^+{} \rightarrow H(p) + H_2
\end{aligned}
\end{equation}

\begin{equation}
\label{eq:emiss6}
\begin{aligned}
\epsilon^{{H}^-{}+{H}^+{}}_{pq} =& A_{pq} PC_{{H}^-{}+{H}^+{}}(T_e,T_{{H}^+{}},T_{{H}^-{}},n_e) n_{{H}^+{}} n_{{H}^-{}} \\
reaction:\ &{H}^-{}+{H}^+{} \rightarrow H(p) + H(1)
\end{aligned}
\end{equation}

The total calculated emissivity and its uncertainty are determined as per \autoref{eq:emiss7} with $\sigma_{ADAS}$ and $\sigma_{Yacora}$ the uncertainty on the coefficients mentioned before.

\begin{equation}
\label{eq:emiss7}
\begin{aligned}
\epsilon^{calc}_{pq} =& \epsilon^{exc}_{pq} + \epsilon^{rec}_{pq} + \epsilon^{{H_2}^+{}}_{pq} + \epsilon^{{H_2}}_{pq} + \epsilon^{{H}^-{}+{H_2}^+{}}_{pq} + \epsilon^{{H}^-{}+{H}^+{}}_{pq}
\\
\sigma^{calc}_{\epsilon_{pq}} =& \left\{ {\sigma_{ADAS}}^2 \left({\epsilon^{exc}_{pq}}^2 + {\epsilon^{rec}_{pq}}^2\right) + \right. \\ &\left. + {\sigma_{Yacora}}^2 \left({\epsilon^{{H_2}^+{}}_{pq}}^2 + {\epsilon^{{H_2}}_{pq}}^2 + {\epsilon^{{H}^-{}+{H_2}^+{}}_{pq}}^2 + {\epsilon^{{H}^-{}+{H}^+{}}_{pq}}^2\right) \right\}^{1/2}
\end{aligned}
\end{equation}

The line emissivity measurement is compared with the expectation. For each precursor combination and line is calculated what is the likelihood that $y_{pq}=0$ with \autoref{eq:emiss8a}
\begin{equation}
\label{eq:emiss8a}
\begin{aligned}
y_{pq} = \epsilon^{calc}_{pq}-\epsilon^{measure}_{pq} ,& \sigma_{y_{pq}} = \sqrt{{\sigma_{\epsilon_{pq}}^{calc}}^2 + {\sigma_{\epsilon_{pq}}^{measure}}^2}
\\
L(y_{pq} = 0|\Theta) =& \frac{1}{\sigma_{y_{pq}} \sqrt{2\pi}} e^{-\frac{1}{2} \left( \frac{y_{pq}}{\sigma_{y_{pq}}} \right)^2 }
\end{aligned}
\end{equation}
where $\Theta$ represent the specific combination of precursors that lead to the emission $\epsilon^{calc}_{pq}$.

Following Bayes theorem the posterior (probability of the combination of precursors generating the measurements) is calculated as the likelihood of the measurement being generated by the precursors times the probability associated with the precursors themselves divided by the probability of the measured data. For the case in which only one emission line is included in the model this is expressed in \autoref{eq:emiss8b}

\begin{equation}
\label{eq:emiss8b}
\begin{aligned}
P(\Theta|y_{pq} = 0) = \frac{L(y_{pq} = 0|\Theta) P(\Theta)}{P(y_{pq} = 0)}
\end{aligned}
\end{equation}

Where $P(y_{pq} = 0)$ acts as a normalisation factor. The final product of all probabilities will be anyway normalised, so this term can be neglected. $P(\Theta)$ is  the product of the probability associated with every combination of precursors (see \autoref{Priors from B2.5 Eunomia}, \ref{Priors from AMJUEL} and \ref{Prior probability distribution}). The probability of fitting all the lines $P_{\epsilon}$ is then determined with \autoref{eq:emiss8}.

\begin{equation}
\label{eq:emiss8}
\begin{aligned}
P_{\epsilon} =& P(\Theta) \prod_{p=4,q=2}^{p=8} P(\Theta|y_{pq} = 0)
\end{aligned}
\end{equation}

For this calculations $\sigma_{ADAS}$ and $\sigma_{Yacora}$ were assumed 10\% and 20\% respectively.

\subsection{Balance over the plasma column}\label{Balance over the plasma column}

\begin{figure*}
	\centering
	\includegraphics[width=\linewidth,trim={0 0 0 0},clip]{Chapters/chapter3/figs/plasma_column.png}
	\caption{Schematic of the plasma column model used. This schematic is useful to correlate local properties (at TS/OES location) to global ones like the total input power. Simplifications as constant densities and temperatures along the magnetic field and constant flow speed are used.}
	\label{fig:plasma_column1}
\end{figure*}

To avoid to consider precursor densities that could well match the line spectra but would lead to unrealistic power or particle losses a balance on the plasma column is performed. The definition of plasma column allows also to extract global information on the ELM-like pulse from the local TS/OES measurements and compare them with other global measurements like the power input from power supply. A schematic of the model of plasma column used is in \autoref{fig:plasma_column1}.

Fundamental assumptions are:
\begin{enumerate}
    \item Given the neutrals density in source and heating chamber is low thanks to differential pumping it is assumed that the plasma is transported undisturbed from the plasma source to the target skimmer. Here $T_e$,$n_e$ are equal to what is measured by TS in the target chamber for the lowest neutral pressure setting, that corresponds to the lowest possible volumetric losses.
    \item The plasma enters the target chamber without any molecular precursor. This is justified by the fact that from source to target chamber skimmer the neutral pressure is at it lowest while the temperature is at its highest and this conditions are the least favourable for reactions involving molecules.
    \item The neutral pressure is fixed throughout the ELM-like pulse to its steady state value.
    \item All plasma properties such as: temperatures, densities, reaction rates, radiated power, depend on radial and temporal coordinates only and are spatially constant from target skimmer to target. This is justified by the fact that the fast camera shows that in Stage 1 and 2 the radiation is mostly constant from a short distance off the target. Given the OES measuring location, only the properties of the bulk of the plasma can be analysed\footnote{Measurements specific to the region close to the target have been attempted but failed possibly due to reflections or obstructions by the target itself.}. That means that the power losses in the visible light brightness peak between the OES and the target observed in \autoref{Fast camera} cannot be accounted, so the volumetric power losses from the analysis will likely be an underestimation. The extent of the non uniform region close to the target, likely including sheath and strongly recycling region, is typically <1cm, small compared to the 38 cm from target to skimmer, so this underestimation should be minor. Increasing neutral pressure from Stage 1 to 2 (the cases we are most interested in) the visible light brightness becomes stronger in the plasma column, making the anisotropy at the target even less relevant. The approximation also neglects\ anisotropy in the visible light brightness in the bulk for very high neutral pressure. This is especially dominant in Stage 3, so it's importance should be minor for Stages 1 and 2.
    \item The plasma behaviour in the sheath and in the strongly recycling region is neglected.
    \item The flow velocity of the plasma is constant from the source to the target.
    \item Cross field transport is negligible (mostly true for charged particles due to the high magnetic field and additionally for molecular ions due to their short life time)
\end{enumerate}

Given these assumptions one can calculate the components of the power and particle balance on the whole plasma column. The OES/TS measurements from a single location can be applied to the whole column and the contribution from atomic and molecular processes can be found.

A quantity that will be used later is the flow velocity ($v_{in}$), the velocity of the plasma in its flow from the target skimmer to the target. It is here mainly used to subdivide the power from the plasma source (a global value) to what is provided to each radial location and to estimate the local particle inflow. It is also used to estimate the kinetic energy of the plasma, but the relevance of this term is minor. There is no direct measurement of $v_{in}$ yet as collective Compton scattering measurements will be available in the future. $v_{in}$ is then approximated by imposing, for the experimental condition with the lowest target chamber neutral pressure, that the power from the source matches the energy flow measured at the TS location. The flow is assumed having a single Mach number for all radial locations. Applying this conditions to \autoref{eq:plasma_column3} this translate to \autoref{eq:plasma_column1} that is then solved to find the Mach number.


\begin{equation}
\label{eq:plasma_column1}
\begin{aligned}
\sum_{r} \biggl\{ \biggl(\frac{1}{2} m_i v_{in}^2(r,t) +
&%\phantom{\biggr) \biggr\}} & \\ \phantom{\biggl\{ \biggl(}+
5k_B T_{e,in}(r,t) + E_{ion} + E_{diss} \biggr) \cdot \phantom{\biggr\}} & \\ \phantom{\biggl\{} \cdot n_{e,in}(r,t) v_{in}(r,t)A \biggr\} &= P_{source}(t)
\\
v_{in}(r,t) &= M_{in}(t) c_{s,up}(r,t) \\  c_s &= \sqrt{\frac{ \left(T_e + T_{H^+{}}( {}\approx T_e) \right)k_B}{m_H} }
\end{aligned}
\end{equation}

In calculating $P_{source}$ as product of voltage and current an efficiency of 92\% is considered in the conversion from electric to plasma energy.\cite{Morgan2014} $M_{in}$ is $\approx$1 during the ELM-like pulse. It must be noted that at the beginning and end of the pulse TS is incapable of accurately measuring across the whole plasma because of the low density, and the energy conversion from electric power to plasma potential is lower, resulting in the calculated $M_{in}>1$. The effect of the overestimation, though, is to allow for larger energy and particle budgets, widening the possible parameter space, so it is acceptable.
I will now detail how to calculate the likelihood associated with the power and particle balance.

\subsection{Power balance}\label{Power balance}

In this chapter it will be detailed how the power (energy) balance equation is obtained and how all the terms are defined.
The 1D energy and particle balance equations are obtained from the 1D Fokker-Planck collisional kinetic equation (\autoref{eq:plasma_column2}) as per derivation from Stangeby\cite{Verhaegh2021} and are adapted using the mentioned approximations for the region from target skimmer to target. This results for every time step and radial location in \autoref{eq:plasma_column3}
\begin{equation}
\label{eq:plasma_column2}
\begin{aligned}
  \frac{ \partial f }{ \partial t}  + v_z \frac{\partial f}{\partial z}  + \frac{eE}{m} \frac{ \partial f }{ \partial v_z} = \left( \frac{ \partial f }{ \partial t} \right)_{coll}  +  S(x,v)
\end{aligned}
\end{equation}
\begin{equation}
\label{eq:plasma_column3}
\begin{aligned}
\frac{  \partial E }{ \partial t} +& \frac{d}{dz}  \left[ \left( \frac{1}{2} m_i v^2 + 5kT_e + E_{ion} + E_{diss} \right) n_e v \right] =\\ -& \underbrace{P_{ext\ source}}_{= 0} + P_{ volume\ sinks-sources }
\\
\frac{  \partial E }{ \partial t} +& \left( \frac{1}{2} m_i v_{in}^2 + 5kT_{e,in} + E_{ion} + E_{diss} \right) n_{e,in} v_{in} =\\ &  + P_{ target } + P_{ volume\ sinks-sources }
\\
P_{diss \: max} =& \underbrace{ \left( \frac{1}{2} m_i v_{in}^2 + 5kT_e + E_{ion} + E_{diss} \right)  {\frac{n_e V}{\Delta t}}}_{P_{ \partial t}} + \\ &+ \underbrace{ \left( \frac{1}{2} m_i v_{in}^2 + 5kT_{e,in} + E_{ion} + E_{diss} \right) n_{e,in} v_{in}A}_{P_{in}} \geq \\ \geq & P_{ volume\ sinks-sources }
\end{aligned}
\end{equation}
with $v$ the flow velocity, $E_{ion}$ and $E_{diss}$ the ionisation and dissociation energy for hydrogen. The inequality arises from not accounting the power delivered to the target and to neglect plasma interactions with neutrals such as elastic collisions and charge exchange. All quantities marked with the subscript ${}_{in}$ refer to the input conditions, otherwise the conditions inside the plasma column are intended. $P_{\partial t}$ represents the power deriving by depleting all the energy associated with the plasma in the volume of interest ($V$) in a single time step ($\Delta t$), $P_{in}$ is the power entering the volume of interest from the plasma source through the area $A$. $P_{diss \: max}$ is the maximum power that can be depleted in a radial portion of the plasma column in a time step. As an additional constrain on the power balance it will be required to $P_{volume\ sink-source}$ to be positive, as otherwise it would mean that the plasma is externally heated on its way to the target.

Let’s investigate now the volumetric sinks-sources term. There are roughly three ways in which a hydrogen plasma can undergo power losses:
\begin{enumerate}
    \item Radiative losses. This mostly comes from the relaxation of excited hydrogen atoms, which can arise from both plasma-atom as well as plasma-molecule interactions. The radiative losses associated with $H_2$ molecular band radiation are expected to be of insignificant.\cite{Groth2019} \label{Radiative losses}
    \item Power transfer from kinetic energy to potential energy. Several plasma species have a relative potential energy associated with it (for instance, $H^+$ has a potential energy of 13.6eV compared to atomic hydrogen). Converting a neutral into an ion thus “converts” 13.6eV of kinetic plasma energy to potential energy.  \label{Power transfer potential}
    \item Power transfer from CX and elastic collisions. CX as well as elastic collisions between the plasma and neutral atoms and molecules can lead to transfer of power from the plasma to the neutrals (and vice versa). This includes collisions between particles of the same specie but at different temperatures like the cold proton generated from ionisation and the hot one part of the plasma. \label{Power transfer CX}
\end{enumerate}

OES combined with collisional radiative models is used to estimate the magnitude of both path \ref{Radiative losses} and \ref{Power transfer potential}, which is employed in this work. For path \ref{Radiative losses}, the Balmer line emission is measured facilitating, through the Bayesian inference of the plasma properties, a full estimate of the hydrogenic line radiation from excited atoms arising both from plasma-atom as well as plasma-molecule interaction. For path \ref{Power transfer potential}, ionisation and recombination rates are estimated to account for the power transfer between potential and kinetic energy. In the recombination reaction a hot $H^+$ is converted into a neutral. That neutral has a kinetic energy equal to the temperature of the plasma that generated it, significantly higher than all other molecular and neutral species. For this reason the energy removed by the plasma assuming the neutrals from recombination escape is accounted in the local power balance. Additionally a series of molecular reactions are considered, see \autoref{Reactions} for which the difference in potential energy between reactants and products is calculated and accounted.

Note that paths \ref{Power transfer potential} and \ref{Power transfer CX} do not strictly represent power lost from the plasma column but can be power transfer mechanisms. Such transfer mechanisms often lead to an effective loss of kinetic energy by the plasma, but can also cause it to increase.

In the power balance that regards the limitation of the power transferred from the plasma at a single radial location, \autoref{eq:plasma_column3}, pathway \ref{Power transfer potential} is considered. It is in fact impossible for the plasma to transfer energy from kinetic to potential for more more than it is available. Differently when the quantity of interest are the components of the global power balance, \autoref{eq:plasma_column9}, only terms where the energy is removed from the plasma column entirely will be considered. Internal power transfer will not be considered as it is energy that remains in the plasma.

Pathway \ref{Power transfer CX} cannot be readily analysed experimentally but can be analysed in detail in simulations. The importance of this is currently discussed in literature and it could be significant especially in low temperature conditions in tokamak divertors and linear machines. \cite{Myatra2021,Smolders2020,Chandra2022} Further code investigations on this and detailed comparisons against experiments are required, which is outside of the scope of this work. To check that neglecting CX and $H_2$ elastic collisions, the ones to have the largest impact\cite{Chandra2022}, does not have a negative impact on the consistency of the solution a crude estimation was done in post processing. This is done by first calculating the ADAS CCD reaction rate for CX and AMJUEL 3.5 rate for $H_2$ elastic. These are multiplied by the density of the reactant species interested and by the maximum energy that can be transfer with a single collision. The energy of the reactants are equivalent to their temperature from TS and \ref{Priors from B2.5 Eunomia}. This results in \autoref{eq:CX_elastic}.
\begin{equation}
\label{eq:CX_elastic}
\begin{aligned}
  P_{CX} &= \frac{ 3 }{ 2}  (T_{H^+}(=T_e)-T_H) RR_{CX}(T_e,n_e) n_{H^+} n_{H}
  \\
  P_{H_2 elastic} &= \frac{ 3 }{ 2} \frac{8}{9} (T_{H^+}-T_{H_2}) RR_{H_2 elastic}(T_{H^+},T_{H_2})  n_{H^+} n_{H_2}
\end{aligned}
\end{equation}
These quantities PDFs are calculated as one of the outputs of the Bayesian algorithm.

%The following is way too complicated. I simply do the product of the rates and that's it
%It is first calculated the inflow of $H_2$ from around the plasma towards the centre assuming sonic flow and the density arsing from the neutral pressure measurement. The consumption of $H_2$ is calculated and the fraction due to CX calculated with ADAS CCD coefficient. The atomic $H$ generated via CX is then assumed flowing outwards carrying the energy corresponding to the plasma temperature. The fraction of this atomic $H$ that undergo a reaction on the way out is then not accounted in the energy loss via CX. It is found that the total CX power loss is negligible respect to other terms. CX is relatively more relevant at low neutral pressure even if the neutral density is lower because the region with high temperature is wider and there is a higher chance of interaction between hot plasma and neutrals.

The sinks/sources terms for \autoref{eq:plasma_column3} are taken from different sources, to encompass the best knowledge available at the time of writing, see \autoref{Reactions}. Grouping them by type and precursor the power balance sinks/sources term is then defined as per \autoref{eq:plasma_column4}

\begin{equation}
\label{eq:plasma_column4}
\begin{aligned}
P_{ \substack{volume \\ sinks-sources}} =& P_{ radiated } + P_{ \substack{neutral\ via \\ recombination} } + P_{ potential\ energy }
\\
P_{ radiated } =& P_{ radiated\ atomic } + P_{ radiated\ molecular } 
\\
P_{ \substack{neutral\ via \\ recombination} } =& \frac{3}{2} \Delta V T_e RR_{rec}
\\
P_{ potential\ energy } =& \Delta V \sum_{i} { {\Delta E}_i \cdot RR_i } 
\\
P_{ radiated\ atomic } =& \underbrace{P_{ excitation }}_{ADAS\ PLT} + \underbrace{P_{ rec + bremsstrahlung }}_{ADAS\ PRB}
\\
P_{ radiated\ molecular } =& P_{ rad\ {H_2}^+{} } + P_{ rad\ {H_2} } + P_{ rad\ {H}^-{}+{H_2}^+{} } + \\ &+ P_{ rad\ {H}^-{}+{H}^+{} } + P_{ rad\ e^-{} + H \rightarrow {H}^-{}+hv } 
\\
P_{ rad,i } =& \Delta V \sum_{p=2,q<p}^{p=13} \epsilon^{i}_{pq}
\end{aligned}
\end{equation}

where $\Delta V$ represent the volume corresponding to the radial location of the plasma considered, $\Delta E_i$ is the energy difference between products and reactants of the reaction $i$ and $RR_i$ is its reaction rate. The ${}^{p=13}$ comes from the fact that only atomic hydrogen excited states up to 13 are here considered. The probability that the inequality in \autoref{eq:plasma_column3} is true and that $P_{volume\ sinks-sources}$ is positive is calculated with \autoref{eq:plasma_column5}
\begin{equation}
\label{eq:plasma_column5}
\begin{aligned}
y =& P_{diss \: max} - P_{s-s}, \sigma_{y} = \sqrt{{\sigma_{P_{\substack{s-s}}}}^2 + {\sigma_{P_{diss \: max}}}^2 }
\\
L_P =& L(y \in [0,P_{diss \: max}]) = \\ =& \frac{1}{2} \left[ erf ( \frac{P_{diss \: max}-y}{ \sqrt{2} \sigma_{y} }) - erf(\frac{-y}{ \sqrt{2} \sigma_{y} } ) \right]
\end{aligned}
\end{equation}
where $P_{volume\ sinks-sources}$ is shortened with $P_{s-s}$.

The power sinks/sources are calculated by adding all the radiative losses to the potential energy contribution. The latter is itself composed by positive and negative contributions that tend to cancel out. This causes the uncertainty of the sinks/sources to greatly dominate over the input one, making the effective use of this balance very difficult. To solve this issue $\sigma_y$ considered as equal to $\sigma_{\substack{P_{diss \: max}}}$ and that is assumed to be 50\% of $P_{diss \: max}$ (using the fixed nominal $T_e$, $n_e$ values from TS). 


\subsection{Particle balance}\label{Particle balance}

The derivation of the particle balance equation from Stangeby\cite{Stangeby2001} results in Equation 14

\begin{equation}
\label{eq:plasma_column6}
\begin{aligned}
\frac{ \partial n_j}{ \partial t} + \frac{d}{dz}   \left(n_j v \right) &= Sinks-Sources \\ (nv)_{j, diss \: max} &={\underbrace{\frac{n_j V }{ \Delta t }}_{(nv)_{j,\partial t}} + \underbrace{n_{j,in} v_{in}}_{(nv)_{j,in}}} \geq \\ \geq & (nv)_{j, Sinks-Sources} = \Delta V \sum_{i} f_{ij} RR_i 
\end{aligned}
\end{equation}

with $f_{ij}$ the multiplicity and sign in the reaction $i$ for the specie $j$ and ${n_{{H_2}}}_{in}={n_{{H}}}_{in}={n_{{H_2}^+}}_{in}={n_{H^-}}_{in} =0$. The inequality comes from not including the particles lost due to surface processes happening at the target and cross field transport. Charged particles are bound by magnetic fields while neutrals can more easily move across. For this reason the particle balance, that considers each radial location independently, is calculated only for charged particles as $e^-$, $H^+$, ${H_2}^+$, $H^-$.
In the case of ${H_2}^+$, $H^-$ the lifetime is very short so even in a single time step it is not physical to allow for its accumulation. For this reason for them $(nv)_{i, Sinks-Sources}$ is limited to be $\geq -(nv)_{i, diss \: max}$. For $e^-$ and $H^+$ the net rate of production is limited to their density the next time step. This term, referred as $(nv)_{j,next}$ is defined similarly to $(nv)_{j,\partial t}$ in \autoref{eq:plasma_column6}. The likelihood that the particle balance is verified is given by \autoref{eq:plasma_column7a} and \ref{eq:plasma_column7b}

\begin{equation}
\label{eq:plasma_column7a}
\begin{aligned}
y_j &= (nv)_{j, diss \: max} - (nv)_{j, Sinks-Sources} \\ {\sigma}_{y_j} &=\sqrt{{{\sigma}_{(nv)_{j, diss \: max}}}^2 + {{\sigma}_{(nv)_{j, Sinks-Sources}}}^2 }
\\
L\left(y_{e^-{}} \in \right. & \left. [0,(nv)_{e^-, diss \: max}+(nv)_{e^-, next}]\right) =\\=& \frac{1}{2} \left[ erf \left(\frac{(nv)_{e^-, diss \: max}+(nv)_{e^-, next}-y_{e^-}}{\sqrt{2} {\sigma}_{y_{e^-}} } \right) \right. +\\ &-\left. erf \left( \frac{-y_{e^-{}}}{\sqrt{2} {\sigma}_{y_{e^-{}}} } \right) \right]
\\
L\left(y_{H^+{}} \in \right. & \left. [0,(nv)_{H^+{}, diss \: max}+(nv)_{H^+{}, next}]\right) =\\=& \frac{1}{2} \left[ erf \left(\frac{(nv)_{H^+{}, diss \: max}+(nv)_{H^+{}, next}-y_{H^+{}}}{\sqrt{2} {\sigma}_{y_{H^+{}}} } \right) \right. +\\ &-\left. erf \left( \frac{-y_{H^+{}}}{\sqrt{2} {\sigma}_{y_{H^+{}}} } \right) \right]
\end{aligned}
\end{equation}

\begin{equation}
\label{eq:plasma_column7b}
\begin{aligned}
L\left(y_{{H_2}^+{}} \in \right. & \left. [0,2(nv)_{{H_2}^+{}, diss \: max}]\right) = \\ =& \frac{1}{2} \left[ erf \left(\frac{2(nv)_{{H_2}^+{}, diss \: max}-y_{{H_2}^+{}}}{\sqrt{2} {\sigma}_{y_{{H_2}^+{}}} } \right) \right. +\\ &-\left. erf \left(\frac{-y_{{H_2}^+{}}}{\sqrt{2} {\sigma}_{y_{{H_2}^+{}}} } \right) \right]
\\
L\left(y_{{H}^-{}} \in \right. & \left. [0,2(nv)_{{H}^-{}, diss \: max}]\right) = \\ =& \frac{1}{2} \left[ erf\left(\frac{2(nv)_{{H}^-{}, diss \: max}-y_{{H}^-{}}}{\sqrt{2} {\sigma}_{y_{{H}^-{}}} } \right) \right. +\\ &- \left. erf\left(\frac{-y_{{H}^-{}}}{\sqrt{2} {\sigma}_{y_{{H}^-{}}} } \right) \right]
\\
L_{nv} &= L(y_{e^-{}} \geq 0) \cdot L(y_{H^+{}} \geq 0) \cdot \\ & \phantom{=} \cdot L(y_{{H_2}^+{}} \in [0,2(nv)_{{H_2}^+{}, diss \: max}]) \cdot \\ & \phantom{=} \cdot L(y_{{H}^-{}} \in [0,2(nv)_{{H}^-{}, diss \: max}]) 
\end{aligned}
\end{equation}

Similarly to what mentioned for the power balance, here too the sinks/sources term is composed of positive and negative factor, so rather than using it a large uncertainty on $(nv)_{i, diss \: max}$ of 100\% for $e$, $H^+$ (using the fixed nominal $n_e$ value from TS) and 5\% of TS $n_e$ for ${H_2}^+$, $H^-$ is assumed. The uncertainties are here adopted so large because differently from power and emissivity there is no direct measurement of the particle input.
As part of the particle balance one has also to include that the density of excited states obtained with ADAS and Yacora coefficients is lower than the density of total atomic hydrogen in the volume. This is calculated with \autoref{eq:plasma_column8}.

\begin{equation}
\label{eq:plasma_column8}
\begin{aligned}
y = n_{H} - \sum_{q,i} n_{i H(q)}, \sigma_{y} = \sqrt{ \sum_{q,i} (\sigma_i n_{i H(q)})^2  }
\\
L_{H^{exc}} = L(y \geq 0) = \frac{1}{2} \left[ 1 - erf\left(\frac{-{y}}{ \sqrt{2} \sigma_{y} } \right) \right]
\end{aligned}
\end{equation}

\subsection{Plasma column power balance}\label{Plasma column power balance}

The definition of the plasma column volume and the power balance allows to evaluate the global performance of the detached target to the ELM-like pulse. The parameter of interest is, in this case, how much power is removed from the plasma column and how much is due to atomic versus molecular effects.
In considering this the potential energy exchange due to EIR, for example, was not considered because it represent a transfer mechanism and not a net loss, while the radiative component due to radiation (ADAS PRB coefficient, returning the losses due to line radiation and Bremsstrahlung) was. Bremsstrahlung radiation is present also in the wavelength range of the IR camera and could be related to the observed prompt emission but this was not investigated. Similarly all other exchanges of potential energy are not considered. The generated neutrals while travelling out of the plasma can react with the neighbouring plasma and the energy they carry be reintroduced. Because evaluating this would require a significant effort this component is for now excluded.

The terms considered for the global power removed from the plasma column are indicated in \autoref{eq:plasma_column9}.
\begin{equation}
\label{eq:plasma_column9}
\begin{aligned}
 E_{ \substack{removed \\ from \\ plasma}} = E_{ radiated } =& \underbrace{E_{exc} + E_{rad\ rec+bremm}}_{atomic} +\\&+ E_{rad\ {H_2} } + E_{rad\ {H}} + E_{rad\ {H_2}^{+{}} } +\\& \underbrace{ +E_{rad\ {H}^{-{}} } + E_{rad\ e^-{}+H \rightarrow {H}^{-{}}+pv \phantom{+}}}_{ molecular }
\end{aligned}
\end{equation}
Once all the likelihoods associated with each combination of priors are calculated they are multiplied to return the total likelihood as per \autoref{eq:plasma_column10}

\begin{equation}
\label{eq:plasma_column10}
\begin{aligned}
L = P_{\epsilon} L_p L_{nv} L_{H^{exc}}
\end{aligned}
\end{equation}

The PDFs of all the components of the power and particle balance previously calculated are built by portioning their range to a smaller number of logarithmic intervals, summing the probability within. For each additive term of interest the PDFs are then convolved in space and in time to obtain the PDF for the whole ELM-like pulse. For each radial and time location and for each output required (for example the total radiated energy) is defined a large list of the possible energy losses in that section of the plasma. The list is randomly distributed according on the PDF of that output at that location. The contribution for all radii is summed to generate a list of possible values of the total radiated energy at one time step. A histogram is built based on that to represent the PDF of the quantity of interest at that time step. This operation is repeated to sum the contribution from all the time steps to return the PDF of the quantity of interest for the whole ELM-like pulse.

\subsection{Reactions}\label{Reactions}

Reaction rates and other coefficients used in the Bayesian calculations are obtained from ADAS\cite{Summers2004,OMullane2013} for the atomic reactions while from Yacora \cite{Wunderlich2016,Wunderlich2020}, AMJUEL\cite{Reiter2017,Reiter2005,Kotov2007} %ALADDIN\cite{IAEANuclearDataSection}
or a collection of reaction rates from Janev\cite{Janev2003} for the molecular reactions. The reactions considered in this work are listed in \autoref{tab:adas}, \ref{tab:yacora}, \ref{tab:amjuel}, %\ref{tab:aladdin}
and \ref{tab:janev}.

\begin{table}[h]
\begin{tabular}{ | p{7cm}| m{3.5cm} | } 
\hline
Reaction: & Chapter / type \\ 
\hline
$H^+ + e^- \rightarrow H(p) + h\nu$ \newline $H^+ + 2e^- \rightarrow H(p) + e^-$ & ADC,PRB,PEC \\ 
\hline
$H + e^- \rightarrow H^+ + 2e^-$ & SCD \\
\hline
$H(q) + e^- \rightarrow H(p>q) + e^-$ & PLT,PEC\\
\hline
\end{tabular}
\caption{Reactions whose rates and reference coefficients were sourced from the ADAS database.\cite{Summers2004,OMullane2013}}
\label{tab:adas}
\end{table}

\begin{table}[h]%[ht!]
\begin{tabular}{ | p{7cm}| m{3.5cm} | } 
\hline
Reaction: & Chapter / type \\ 
\hline
${H_2}^+ + H^- \rightarrow H(p) + H_2$ &  \\
\hline
$H^+ + H^- \rightarrow H(p) + H$ & \\
\hline
${H_2}^+ + e^- \rightarrow H(p) + H(1)$ & \\
\hline
${H_2}^+ + e^- \rightarrow H(p) + H^+ + e^-$ & \\
\hline
$H_2 + e^- \rightarrow H(p) + H(1) + e$ & \\
\hline
\end{tabular}
\caption{Reactions whose rates and reference coefficients were sourced using the Yacora collisional radiative code.\cite{Wunderlich2016,Wunderlich2020}}
\label{tab:yacora}
\end{table}

\begin{table}[h]
\begin{tabular}{ | p{7cm}| m{3.5cm} | } 
\hline
Reaction: & Chapter / type \\ 
\hline
$e^- + H_2 \rightarrow 2e^- + {H_2}^+$ & 4.11 Reaction 2.2.9 \\
\hline
$e^- + H_2 \rightarrow 2e^- + H + H^+$ & 4.12 Reaction 2.2.10 \\
\hline
$e^- + {H_2}^+ \rightarrow 2e^- + H^+ + H^+$ & 4.13 Reaction 2.2.11 \\
\hline
$e^- + {H_2}^+ \rightarrow e^- + H + H^+$ & 4.14 Reaction 2.2.12 \\
\hline
$e^- + {H_2}^+ \rightarrow H + H$ & 4.15 Reaction 2.2.14 \\
\hline
$e^- + H_2 \rightarrow e^- + H_2(v) \rightarrow H + H^-$ & 2.23 Reaction 2.2.17 \\
\hline
$H^+ + H_2(v) \rightarrow H + {H_2}^+$ & 3.28 Reaction 3.2.3 \\
\hline
$H^+ + H^- \rightarrow H + H$ & 4.52 Reaction 7.2.3a \\    % this one CANNOT be deleted as a reaction in Yacora too
\hline
$e^- + H_2(v) \rightarrow e^- + H + H$ & 4.10 Reaction 2.2.5g \\    % this one CANNOT be deleted as a reaction in Yacora too
\hline
\end{tabular}
\caption{Reactions whose rates and reference coefficients were sourced from the AMJUEL database.\cite{Reiter2017,Reiter2005,Kotov2007}}
\label{tab:amjuel}
\end{table}

% \begin{table}
% \begin{tabular}{ | p{5.5cm}| m{2.8cm} | } 
% \hline
% Reaction: & Chapter / type \\ 
% \hline
% neglected because very uncertain rate
%${H_2}^+(1s\sigma g;v=0-9) + H(1) \rightarrow $\newline $\rightarrow H^+ + H(1) + H(1)$ & Reaction rate \\
%\hline
% neglected because very uncertain rate
%$H(1) + H(3) \rightarrow {H_2}^+(v) + e^-$ & Cross section \\
%\hline
% neglected because very uncertain rate
%$e^- + H^- \rightarrow e^- + H + e^-$ & Reaction rate \\
%\hline
% neglected because very uncertain rate
%$H^- + H_2(v) \rightarrow H + H_2(v") + e^-$ & Cross section \\
% \hline
% \end{tabular}
% \caption{Reactions whose rates and reference coefficients were sourced from the iaea.org/ALADDIN database}
% \label{tab:aladdin}
% \end{table}

\begin{table}[h]
\begin{tabular}{ | p{7cm}| m{3.5cm} | } 
\hline
Reaction: & Chapter / type \\ 
\hline
% neglected because very uncertain rate
%${H_2}^+(vi) + H_2(v0) \rightarrow ({H_3}^+* + H) \rightarrow $\newline $\rightarrow H^+ + H + H_2(v01)$ \newline ${H_2}^+(vi) + H_2(v0) \rightarrow $\newline $\rightarrow [{H_2}^+(2p\sigma u/2p\Pi u···) + H_2] \rightarrow $\newline $\rightarrow H^+ + H + H_2(v01)$ & 7.3.2 \\
%\hline
% neglected because very uncertain rate
%$H^+ + H^- \rightarrow {H_2}^+(v) + e^-$ & 3.2.2 \\
%\hline
% neglected because very uncertain rate
%$H^+ + H(1) + H(1) \rightarrow H^+ + H_2(\nu)$ & 2.2.4, equation 46a \\
%\hline
% neglected because very uncertain rate
%$H^+ + H(1) + H(1) \rightarrow H(1) + {H_2}^+(\nu)$ & 2.2.4, equation 46b \\
%\hline
% neglected because very uncertain rate
%$H(1) + H(2) \rightarrow {H_2}^+(v) + e^-$ & 2.3.3 \\
%\hline
% neglected because very uncertain rate
%$H(1) + H(4) \rightarrow {H_2}^+(v) + e^-$ & 2.3.3 \\
%\hline
% neglected because very uncertain rate
%$H+H+H \rightarrow H_2 + H$ & 2.3.4 \\
%\hline
% neglected because very uncertain rate
%$H^- + H(1) \rightarrow {H_2}^-(B2\Sigma+g) \rightarrow $\newline $\rightarrow H(1) + H(1) + e^-$ & 3.3.2 \\
%\hline
% neglected because very uncertain rate
%$H^- + H(1) \rightarrow {H_2}^-(X2\Sigma+u;B2\Sigma+g) \rightarrow $\newline $\rightarrow {H_2}(X1\Sigma+g,v) + e^-$ & 3.3.2 \\
% neglected because very uncertain rate
%$H^+ + H_2(X1\Sigma+g,v) \rightarrow H^+ + H(1) + H(1)$ & 5.3 \\
% neglected because very uncertain rate
%$H(1) + H_2(v) \rightarrow H(1) + H(1) + H(1)$ & 6.2.2 \\
%\hline
% neglected because very uncertain rate
%$H_2(v=0) + H_2(v) \rightarrow H_2(v=0) + H(1) + H(1)$ & 6.3.2 \\
%\hline
% neglected because very uncertain rate
%$e^- + H(1) \rightarrow H^- + h\nu$ & 2.1.3 \\
%\hline
${H_2}^+(vi) + H^- \rightarrow ({H_3}^*) \rightarrow $\newline $\rightarrow H_2(X1\Sigma_g;v_0) + H(n\geq 2)$ & 7.4.1 \\    % this one CANNOT be deleted as a reaction in Yacora too
\hline
${H_2}^+(vi) + H^- \rightarrow ({H_3}^*) \rightarrow $\newline $\rightarrow H_2(N1,3\Lambda_{\sigma};v_0) + H(1),N \leq 4$ & 7.4.1 \\   % this one CANNOT be deleted as a reaction in Yacora too
\hline
\end{tabular}
\caption{Reactions whose rates and reference coefficients were sourced from the Janev database.\cite{Janev2003}}
\label{tab:janev}
\end{table}





% Some people include this, you don't have to but a bit of self promotion never hurt!
%\chapter{List of publications and communications}
%\input{Chapters/Publications/Publications}

% References
\phantomsection
\addcontentsline{toc}{chapter}{List of References}  % Add references to list of contents

% Bibliography style
\bibliographystyle{yorkThesisBibStyle}
\bibliography{references} % .bib file that contains your references (use Jabref, Mendeley, or Endnote etc to manage your references)

\end{document}