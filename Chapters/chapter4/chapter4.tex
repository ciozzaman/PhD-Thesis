

In this work different aspects of detachment have been investigated. In \autoref{chapter2} a new diagnostic was developed for the MAST-U spherical tokamak (IRVB) to study the changes in total emissivity distribution depending on the plasma conditions and level of detachment. In \autoref{chapter2.5} an algorithm for tomographic reconstruction was developed and applied to the IRVB data to further examine the change in the radiated power distribution in the divertor. In \autoref{chapter3} the resilience to ELM-like pulses of a detached plasma produced in a linear device was analysed, in order to understand the processes involved with the burn through, as part of a collaboration with DIFFER.

\autoref{chapter2} describes the steps that were needed to develop the IRVB diagnostic for MAST-U. The goal of the diagnostic is to monitor the changes of total radiated power around the x-point, a region where strong gradients of emissivity are expected, and relate them to detachment. Focussing on the x-point also allows us to fill the gap between the core and SXD chamber resistive bolometry systems. The IRVB original geometry optimisation and hardware is designed by Matthew Reinke, based on previous work for NSTX-U. \cite{VanEden2016}

The design for the first MAST-U experimental campaign featured a large FOV so as to almost reach the mid-plane, including both a poloidal and a tangential view of the plasma. The design was verified using the ray tracing code CHERAB and was optimised to have the best spatial resolution in the defined region of interest while maintaining a sufficient SNR. The FOV was verified, a particularly difficult task for foil bolometers, by using data from MU01. During disruptions the increase of foil temperature highlights the silhouette of the central column, while fuel locations cause strong localised emissivity. In both cases, their location on the foil matched the expectation from design. The observed changes in brightness was also well matched with the location of the separatrix from EFIT.

In order to obtain the power absorbed by the foil, the system was fully characterised in the conversion from camera counts to foil temperature and foil power. In this process the uncertainties were estimated so that a full propagation of the errors could be performed and used in later inversions. An ad hoc desktop setup was built in order to characterise the foil in the same conditions as in the experiments.

The resulting brightness shows the movement along the separatrix of a region with high emissivity, from the inner target to the x-point, and then to the HFS midplane, related to detachment, matching expectations. The changes of brightness are compared with the resistive bolometry system and the two show similar trends, validating the IRVB diagnostic. The noise levels are shown to be better than expectations, with the largest uncertainty due to the variability of the foil and coating properties. This is the first implementation of an IRVB diagnostic in a spherical tokamak and its operation is here validated in MAST-U.



In parallel to the hardware work, an inversion routine to convert the line integrated data to local emissivity was developed, as shown in \autoref{chapter2.5}. Various methodologies were considered and a Bayesian method was implemented to make full use of the previous characterisation of the uncertainty, to compensate for the foil non uniformity, and to adapt to the changing signal strength. The inversion algorithm was checked against a more established method, SART, returning similar or better results.

The results from MU01, generally affected by low shot repeatability, show that for low power L-mode discharges, the movement of the radiator is as would be expected from ordinary large aspect ratio tokamaks, with the inner leg detaching first and then the outer. These changes are related to the particle flux roll over measured by LPs (the onset of detachment) with radiative detachment on the outer leg happening at about the same time, or slightly earlier. Once the region of high brightness reaches the x-point, it then moves upstream following the HFS separatrix up to the midplane, forming a MARFE-like structure that can then move into the core. This structure is often elongated along field lines and seems to be related to the type of fuelling applied to the discharge being more localised with HFS fuelling. The integrated emissivity is in agreement with resistive bolometry measurements and shows a radiated power fraction reaching $\sim$50\% at the emergence of the MARFE-like structure. This progression of radiative detachment with upstream density is confirmed across multiple shots of this type. Radiation on the inner leg is observed to gradually move from the target to the x-point. This is different from expectations from theory that, because the movement is in the same direction as the toroidal field gradient, suggest an unstable transition. The radiation and level of detachment is observed not to be significantly affected by $\delta R_{sep}$, a metric of how close the plasma geometry is to a single or double null, contrary to what is observed in other experiments where it can significantly impact up/down power sharing for changes of the order of $\lambda_q$.

High power NBI heated H-mode discharges were performed, but were generally even more affected by repeatability issues than L-mode ones, meaning that a set of similar discharges to compare the behaviour for different stages of detachment could not be found. The results for one shot were shown, where detachment is caused by increased fuelling from the lower divertor. Given the non up/down symmetric fuelling, the radiation profile was also not symmetric, so the total radiated power cannot be inferred by only observing the lower half of the machine. Radiative detachment first starts at the outer target, then stops at an intermediate level, the inner target then fully detaches and, finally, the outer leg fully detaches. The particle flux roll over is observed on the lower outer target at about the same time as the full radiative detachment is achieved. After the radiation on both legs has detached up to the x-point, it moves upstream on the separatrix to form the MARFE-like structure on the HFS midplane. Given the higher power injected in the plasma, a much higher density can be reached of up to 70\% of the Greenwald fraction. As mentioned before, the upper and lower divertors behave differently, with the lower reaching detachment while the upper is still attached. The integrated radiated power, accounting for the asymmetry, likely peaks at about 60\% of the input power, showing that even in H-mode, and without extrinsic impurities, a large fraction of the input power can be dissipated via radiation. The different behaviour of the lower and upper divertors could be due to the baffles that allow for a good neutral compression to be achieved between the divertors and the main chamber. This may allow independent control of the two divertors.

Because only a small fraction of the IRVB LOSs enter the SXD chamber, and the solid angle between them is small, it was previously demonstrated that it is not possible to accurately reconstruct the radiation distribution inside. This prevents, with the current IRVB geometry, characterisation of the progress of radiative detachment for super-x plasmas. The IRVB can, anyway, determine the integrated power emitted in the SXD chamber with sufficient precision. The total radiated power in the SXD chamber compared to the rest of the outer leg was used, in conjunction with data from the multi wave imaging diagnostic (MWI), to infer the atomic and molecular processes that are dominant for increasing levels of detachment. Given the relatively low power available in MU01 in the super-x configuration (no NBI, L-mode), the outer target at the beginning of the shot is already partially detached, showing the ease of achieving detachment in this configuration. Most of the radiation on the outer leg, in the early stages of detachment, is inferred to be due to hydrogen electron impact excitation (related to ionisation), rather than carbon line emission. For increasing density the importance of molecular processes increases and then electron ion recombination starts growing.


\autoref{chapter3} shows the work regarding the resilience of the detachment front to ELMs. ELM-like pulses were reproduced in the linear plasma machine Magnum-PSI with the use of a dedicated capacitor bank in parallel to the steady state power source supply. The ELM-like pulses increase the power to the plasma source for $\sim$0.5ms by about 5 times, with a similar increase of the heat flux through the plasma and to the target. Thanks to differential pumping among three separated chambers, it was possible to maintain the same upstream conditions while increasing the neutral pressure in the target chamber in order to cause detachment of the steady state plasma. It is observed that 3 distinct stages of the burn through develop. In stage 1 the plasma is attached in steady state and during the ELM-like pulse, the interactions in the volume are minor and a high heat flux is deposited on the target. Increasing the neutral pressure, the plasma transitions to stage 2, detaching in steady state and reattaching during the ELM-like pulse. The heat flux to the target is reduced but still significant, while the increase of H$\alpha$ emission observed is characteristic of a colder plasma. Increasing the pressure further the plasma reaches stage 3, in which the ELM-like pulse energy is dissipated entirely in the volume and no significant heat is transferred to the target. In the transition from stage 1 to 2, the heat flux factor, a metric used to assess the effect of ELMs on the target, is reduced by about half while it becomes negligible in stage 3. In a tokamak the connection length is of the order of tens of meters on the outer target and the divertor neutral pressure can reach the Pascal range, while in these experiments the connection length in the target chamber is 0.38m and the pressure up to 15Pa. Considering that, if the divertor is baffled, the neutral pressure can be further increased by more than one order of magnitude, these results imply that a significant reduction of the heat flux factor in tokamaks should be possible thanks to the high neutral density region forming close to the target when it is strongly detached, improving the long term performance of the target in H-mode. 

The existing optical emission spectroscopy diagnostic, observing the plasma in the target chamber, was upgraded to increase the time resolution and perform intra-ELM measurements. The diagnostic has a series of LOS observing the plasma tangentially so that the local line emissivity can be inferred assuming poloidal symmetry. From the change in the line ratio, the importance of atomic (ionisation, recombination, dissociation) and molecular (reactions involving the plasma and neutral interactions with ${H_2}^+$ and $H^-$ precursors) processes can be assessed using coefficients from collisional radiative models. The setup did not allow the measurement of all the hydrogen Balmer lines so only the lines $p=4-\infty \rightarrow 2$ could be measured and of those $p=4-8 \rightarrow 2$ had a sufficient SNR. Not including H$\alpha$ implied that the analysis of the line ratios alone could not help in distinguishing between the effect of ${H_2}^+$ and $H^-$. In order to improve the physical consistency of the results, a Bayesian algorithm and a very crude model of the plasma column were developed to combine the data from different diagnostics and limit the parameter space of the variables considered. The results from this analysis show that, as expected, the type of process that dominates in the plasma depends mainly on the plasma temperature. For temperatures above 5eV ionisation and dissociation, both via electron impact and molecular precursors, dominate. At lower temperatures molecular reactions are more important2, and below 1.5eV electron ion recombination dominates. The power losses due to radiation account for $\sim$1/3 of the input power at high pressure, and 1/3 of this portion is due to molecular reactions. Most of the thermal energy of the plasma is exchanged to potential energy. At low pressure the plasma generated via ionisation is more than from the plasma source, a unique case in linear machines. At higher pressure recombination, both via molecules ($\sim$1/3 of the total) and EIR, become dominant such that the particle flux to the target is greatly reduced. The impact of molecular reactions on the radiative losses is significant but not dominant while, as observed in previous research, molecular activated recombination and especially dissociation play an important role in the power balance. This implies that if the burn through process with a strongly detached target has to be modelled, or the (low temperature) far SOL has to be accounted for, the inclusion of molecular reactions is required to capture the behaviour at $5eV>T_e>1.5eV$ and to estimate $H_2$ dissociation correctly.

\section{Outlook}\label{Outlook}


The design, calibration and verification of the IRVB diagnostic, still classified as a prototype in MAST-U, has allowed best practices and improvements for future implementations to be found, such as:

\begin{itemize}
    \item Orient the camera and viewport such to avoid self reflection
    \item Perform the temperature calibration of the IR camera with a large black body source encompassing the whole FOV
    \item Use a foil with the coating applied via chemical vapour deposition rather than sprayed to reduce its non uniformity
    \item Aim the FOV in the co-NBI direction to prevent neutrals from charge exchange to enter through the pinhole and effect measurements
\end{itemize}

Given the lower than expected noise level observed in MU01 the IRVB geometry was modified ahead of MU02 to concentrate the FOV even more around the x-point and divertor. This should provide an even greater spatial resolution, and possibly allow better assessment of the penetration of the radiator inside the core and the radiation distribution in the SXD chamber.

It is also planned to add a new IRVB aimed at the upper x-point. Together with an additional resistive bolometry system for the upper divertor and the refinement of the current core and lower divertor resistive systems, this will allow monitoring of the radiated power profile in the entire machine. This is necessary for power balance studies and will allow a full characterisation of the evolution of radiation with detachment for all divertor configurations possible in MAST-U.

Further improvements can also come from better data analysis, like implementing the Empirical Bayes method for the simultaneous estimation of regularisation and negative correction coefficients. Another avenue is to try merging the resistive system and IRVBs to perform a tomographic inversion of the entirety of the MAST-U plasma. This could return even more accurate estimates of the radiated power.

Regarding the study of ELM burn through, implementing a spectroscopic setup capable of measuring intra-ELM line emission in tokamaks is very challenging, so a better understanding could still come from linear machines. The crude plasma column model detailed in this work could be improved allowing a gradual change of the plasma properties along the plasma column. This could be achieved by defining an analytical profile shape for parameters like $n_e$, $T_e$, $n_{H_2}$, $n_{H}$, $n_{{H_2}^+}/n_{H_2}$, $n_{{H}^-}/n_{H_2}$ and then finding the most likely values for the parameters of the analytical expressions in a similar fashion as done in \cite{Gahle2020}. This could be coupled with a model for the neutrals capable of accounting for transport such as Eunomia. Additionally, other experimental data like the total radiated power from bolometry and the flow velocity from collective Thomson scattering (which was later added to the Magnum-PSI diagnostic suite) could be added to the Bayesian analysis.

It might also be important to repeat the experiment with extrinsic impurities, to establish the impact on the radiated power losses and if they facilitate the ELM baffling.

