

\section{MAST-U and the IRVB diagnostic}\label{MAST-U and the IRVB diagnostic}
\hl{to do}

\section{ELM-like pulses in Magnum-PSI}\label{ELM-like pulses in Magnum-PSI}
The effect of ELM-like pulses on a detached target in Magnum-PSI was studied with the help of various diagnostics. It was found that the ELM-like pulse energy can be effectively dissipated for high level of detachment before it reaches the target. The decreasing power reaching the target seems related to higher volumetric power losses in the volume between target and source. They seem initially fairly uniform becoming more and more uneven increasing the target chamber neutral pressure further, correlated with increased loss of plasma before reaching the location where it is measured. This behaviour can be divided in stages. In Stage 1 the plasma is attached before and during the ELM-like pulse and the energy losses in the volume are at a minimum and predominantly not radiative. In Stage 2 the target chamber neutral pressure is such that the plasma is detached from the target in steady state but can reattach during the ELM-like pulse. Radiative energy losses in the volume increase to dominate over other mechanisms, mainly driven by molecular assisted reactions. In Stage 3 the plasma is detached before and during the pulse and the losses in the volume are such that the plasma cannot reach the target, being effectively baffled.

Hydrogen Balmer line emission from optical emission spectroscopy was used to develop a Bayesian routine that incorporates organically the results from multiple diagnostics to return the power balance in the plasma column in the target chamber. Various approximations to extrapolate the results from a single location in the target chamber to its entirety were adopted. The routine incorporates priors from numerical simulations and collisional radiative codes so that atomic interactions can be distinguished from molecular ones. It was found that radiation from the plasma due to molecular assisted reactions is an important but not dominant energy loss mechanism. Mutual neutralisation of ${H}^-$ seems to dominate radiative losses, but it was established that this cannot be determined from OES alone with the present setup. Molecular assisted reactions significantly effect the local power balance via the exchange of potential energy, limiting what is available for ionisation. MAD is the most significant path via molecular precursors, but this could be due to an unconstrained paticle balance for $H_2$ and $H$. Molecular processes are mostly dominant in an intermediate temperature range around 3eV, between ionisation and EIR dominant regions.

These results indicate that for highly detached regimes in linear machines, and most likely also Tokamaks, molecular interactions are important and need to be accounted both in terms of potential energy exchange and induced hydrogenic line radiation, something that it’s not yet fully done in many codes used for Tokamak edge plasma studies.
