
\chapter{Overall summary}\label{chapter3.9}


In this work different aspects of detachment have been investigated. A new diagnostic was developed for the MAST-U spherical tokamak (IRVB) to study the changes in total emissivity distribution depending on the plasma conditions and level of detachment. An algorithm for tomographic reconstruction was developed and applied to the IRVB data to further examine the change in the radiated power distribution in the divertor. The resilience to ELM-like pulses of a detached plasma produced in a linear device was analysed, in order to understand the processes involved with burn through, as part of a collaboration with DIFFER.

\section{MAST-U}



The activities in MAST-U are aimed at characterising radiative detachment, which occurs in parallel to detachment as observed from particle flux roll over, for various experimental conditions in a spherical tokamak. This is important as the spherical tokamak configuration is expected to allow reaching higher plasma pressure than large aspect ratio tokamaks for the same magnetic field strength. A disadvantage is that the available target area to absorb the exhaust from the core is smaller, so heat management becomes even more important. It is rare in large aspect ratio tokamaks to have a diagnostic with such a high spatial resolution in the divertor as in this work (IRVBs are usually deployed to monitor core radiation) and it is unique for spherical tokamaks.



\autoref{chapter2} describes the steps to develop the IRVB diagnostic for MAST-U. The goal of the IRVB is to monitor the changes of total radiated power around the x-point, a region where strong gradients of emissivity are expected, and relate them to detachment. Focusing on the x-point also allows us to fill the gap between the core and SXD chamber resistive bolometry systems. The IRVB original geometry optimisation and hardware is designed by Matthew Reinke, based on previous work for NSTX-U. \cite{VanEden2016}

The design for the first MAST-U experimental campaign featured a large FOV so as to almost reach the HFS midplane, including both a poloidal and a tangential view of the plasma. The design was optimised using the ray tracing code CHERAB to have the best spatial resolution around the x-point while maintaining a sufficient SNR. The FOV was verified, a particularly difficult task for foil bolometers, by using the plasma from MU01 itself as a light source. During disruptions the increase of foil temperature highlights the silhouette of the central column, while fuel locations cause strong localised emissivity. In both cases, their location on the foil matched the expectation from design. The observed changes in brightness was also well matched with the location of the separatrix from EFIT. This is particularly relevant in situations where in vessel access is restricted or a full characterisation of the field of view would be too onerous.


To obtain the power absorbed by the foil, the system was fully characterised, from camera counts to foil temperature and foil power. In this process the uncertainties were estimated so that a full propagation of the errors could be performed and used in later analysis. An ad hoc vacuum bench top setup was built in order to characterise the foil.



The resulting brightness shows the movement along the separatrix of a region with high emissivity, from the inner target to the x-point, and then to the HFS midplane, that is expected from detachment. The changes of brightness are compared with the resistive bolometry system and the two show similar trends, validating the IRVB diagnostic.

The noise levels are shown to be better than expectations, with the largest uncertainty due to the variability of the foil and coating properties.



In parallel to the hardware work, an inversion routine to convert the line integrated data to local emissivity was developed, as shown in \autoref{chapter2.5}. Various methodologies were considered and a Bayesian method was implemented to make full use of the previous characterisation of the uncertainty, to compensate for the foil non uniformity, and to adapt to the changing signal strength. The inversion algorithm was checked against a more established method, SART, returning similar or better results.


The results from MU01 show that for low power L-mode discharges the movement of the radiator is as would be expected from ordinary large aspect ratio tokamaks, with the inner leg detaching before the outer. These changes are related to the particle flux roll over measured by LPs (the onset of detachment) with radiative detachment on the outer leg happening at about the same time, or slightly earlier. Once the region of high emissivity reaches the x-point, it then moves upstream following the HFS separatrix up to the midplane, forming a MARFE-like structure localised on the HFS midplane that can then move into the core. This structure is often elongated along flux surfaces, being more localised and less elongated with HFS fuelling.

The integrated emissivities (radiated power) in the regions in which MAST-U is divided are in agreement with resistive bolometry measurements, and show a radiated power fraction reaching $\sim$50\% after the emergence of the MARFE-like emissivity peak. This progression of radiative detachment for increasing upstream density is confirmed across multiple shots of this type.

Radiation on the inner leg is observed to gradually move from the target to the x-point. This is different from expectations from theory that, because the movement is in the same direction as the toroidal field gradient, suggests a rapid jump with no solutions in between. This measurement was possible only because of the high spatial resolution achieved.

The radiation and level of detachment is observed not to be significantly affected by $\delta R_{sep}$, a metric of how close the plasma geometry is to a single or double null, contrary to what is observed in other experiments where it can significantly impact up/down power sharing for changes of the order of $\lambda_q$. The reason for this is inferred from SOLPS simulations to be a relatively low plasma current in MU01 compared to the MAST-U design (I$\leq$750kA instead of 2MA), limiting the influence of drifts.

High power NBI heated H-mode discharges were performed but the shots were less successful, so a set of similar discharges to compare the behaviour for different stages of detachment could not be found. The results for one shot were shown, where detachment is caused by increased fuelling from the lower divertor. Given the non up/down symmetric fuelling, the up/down radiation balance was also not symmetric, so the total radiated power cannot be inferred by only observing the lower half of the machine. Due to the higher power being radiated the inversion algorithm returned emissivity profiles with a higher spatial resolution.

Radiative detachment first starts at the outer target, then stops at an intermediate location between target and x-point, then the inner target then fully detaches and finally, the outer leg fully detaches. The particle flux roll over is observed on the lower outer target at about the same time as full radiative detachment is achieved. After the radiation on both legs has detached up to the x-point, the radiation peak moves upstream on the HFS separatrix to form the MARFE-like structure on the HFS midplane. 

Given the higher power injected in the plasma in H-mode, a much higher density can be reached compared to L-mode before incurring disruptions, increasing the peak Greenwald fraction from 30\% up to 70\%. As mentioned before, the upper and lower divertors behave differently due to asymmetric fuelling, with the lower reaching detachment while the upper is still attached. This different behaviour could be due to the baffles. They enhance neutral compression between the divertors and the main chamber, limiting the equalisation of the neutrals between the volumes. This might suggest the possibility to independently control the two divertors.

The total radiated power, accounting for the asymmetry, likely peaks at about 60\% of the input power, showing that even in H-mode, and without extrinsic impurities, a large fraction of the input power can be dissipated via radiation.


In MU01 only a small fraction of the IRVB lines of sight enter the SXD chamber, and the solid angle between them is small. It was demonstrated that it is not therefore possible to accurately reconstruct the radiation distribution in the SXD chamber. This prevents, with the current IRVB geometry, the characterisation of the progress of radiative detachment in the SXD chamber, needed for super-x plasmas. It is still possible to use the IRVB to determine the integrated power radiated in the SXD chamber with sufficient precision. This, compared to the power radiated in the rest of the outer leg, was used in conjunction with data from the multi wave imaging diagnostic (MWI), to infer the atomic and molecular processes that dominate the IRVB-measured radiation for increasing levels of detachment.

Given the relatively low power available in MU01 in the super-x configuration (no NBI, L-mode), the outer target at the beginning of the shot is already partially detached, showing the ease of achieving detachment in this configuration. Most of the radiation from the outer leg, in the early stages of detachment, is inferred to be due to hydrogen electron impact excitation (related to ionisation), rather than carbon line emission. For decreasing temperature in the SXD chamber the importance of molecular processes increases and then at even lower temperatures EIR becomes dominant.
This implies that molecular processes play a role in deep detachment and, if accurate simulations are needed, they have to be included, especially when the temperature decreases below $\sim$5eV.





\section{Magnum-PSI}

\autoref{chapter3} shows the work regarding the resilience of the detachment to ELMs. ELM-like pulses were reproduced in the linear plasma machine Magnum-PSI with the use of a dedicated capacitor bank in parallel to the steady state power source supply. The ELM-like pulses increase the power to the plasma source for $\sim$0.5ms by about 5 times, with a similar increase of the heat flux through the plasma and to the target. Thanks to differential pumping among three separated chambers, it was possible to maintain the same upstream conditions while increasing the neutral pressure in the target chamber in order to cause detachment of the steady state plasma.

It is observed that 3 distinct stages of the burn through develop. In stage 1 the plasma is attached in steady state and during the ELM-like pulse, the interactions in the volume are minor and a high heat flux is deposited on the target. Increasing the neutral pressure, the plasma transitions to stage 2, detaching in steady state and reattaching during the ELM-like pulse. The heat flux to the target is reduced but still significant, while the increase of Balmer $\alpha$ emission observed is characteristic of a colder plasma. Increasing the pressure further the plasma reaches stage 3, in which the ELM-like pulse energy is dissipated entirely in the volume and no significant heat is transferred to the target.

In the transition from stage 1 to 2, the target heat flux factor, a metric used to assess the effect of ELMs on the target, is reduced by about half, becoming negligible in stage 3. In a tokamak the connection length from the ionisation front to the outer target can be (in deep detachment) of the order of meters and the divertor neutral pressure can reach the Pascal range, while in these experiments the connection length in the target chamber was 0.38m and the pressure up to 15Pa. If divertor neutral compression is enhanced by baffling, the divertor neutral pressure can be increased by more than one order of magnitude compared to the main chamber, limiting the negative effects on the core due to a high neutral pressure. These results imply that a significant reduction of the heat flux factor in tokamaks could be possible thanks to the high neutral density region forming close to the target when it is strongly detached, improving the long term performance of the target in H-mode. 

The existing optical emission spectroscopy diagnostic, observing the plasma in the target chamber, was upgraded to increase the time resolution and perform intra-ELM measurements by the author and Gijs Akkermans. The diagnostic has a series of LOS observing the plasma tangentially so that the local line emissivity can be inferred assuming poloidal symmetry. From the change in the hydrogen Balmer line emission ratio, the importance of atomic processes (ionisation, recombination, dissociation) and molecular processes (reactions involving the plasma and neutral interactions with ${H_2}^+$ and $H^-$ precursors) can be assessed using coefficients from collisional radiative models. The setup did not allow the measurement H$\alpha$, implying that the analysis of the line ratios alone could not help in distinguishing between the effect of ${H_2}^+$ and $H^-$. In order to improve the physical consistency of the results, a Bayesian algorithm and a very crude model of the plasma column were developed to combine the data from different diagnostics and limit the parameter space of the variables considered.

The results from this analysis show that, as expected, the type of process that dominates in the plasma depends mainly on the plasma temperature. For temperatures above $\sim$5eV, ionisation and dissociation (both via electron impact and molecular precursors) dominate. At lower temperatures molecular reactions are more important, and below $\sim$1.5eV electron ion recombination dominates. The power losses due to radiation account for $\sim$1/3 of the input power at high pressure, and 1/3 of this portion is due to molecular reactions.

Most of the thermal energy of the plasma is exchanged to potential energy. At low pressure, the plasma generated via ionisation during the part of the ELM-like pulse with the higher temperature is more than that produced by the plasma source, a unique case in linear machines. At higher pressure recombination, both via molecules ($\sim$1/3 of the total) and EIR, dominates particle and power balance, and the particle flux to the target is greatly reduced. This is in accordance with direct heat flux measurements. 

The impact of molecular reactions on the radiative losses is significant but not dominant while, as observed in previous research, molecular activated recombination and especially dissociation play an important role in the power balance. This implies that if the burn through process with a strongly detached target has to be modelled, or the (low temperature) far SOL has to be accounted for, the inclusion of molecular reactions is required to capture the behaviour at $5eV>T_e>1.5eV$ and to estimate $H_2$ dissociation correctly.


\chapter{Conclusions}\label{chapter4}

The aim of this work is a better understanding of the progression and processes involved with divertor detachment. %They are split between activities regarding MAST-U and Magnum-PSI.

\section{MAST-U}

The activities in MAST-U were aimed at characterising the role and dynamics of total radiation during detachment in a spherical tokamak. This was made possible with the development of the IRVB diagnostic. The IRVB field of view (FOV) is aimed at the x-point, and the diagnostic was used to reconstruct the radiation profile in the lower half of MAST-U. With a single IRVB, a spatial resolution of the order of few cm was achieved, a level difficult to achieve with resistive bolometers, allowing for clear differentiation of the radiative detachment of the inner from the outer leg. The IRVB could also be reactor relevant, as the only component significantly exposed to neutron irradiation is the foil, contrary to the resistive bolometer where some elements are glued to the back of the foil. The IRVB foil characteristics can be adjusted to reduce its degradation over time, and the camera can be located in a safer location looking through a periscope system.

The high IRVB spatial resolution allowed monitoring of the radiation profile on the inner leg. It was found that radiative detachment happens gradually, with the radiator gradually moving from the target to the x-point, appearing to be contrary to expectations from analytical models of detachment stability. This has beneficial implications for controlling the heat flux to the target.

Radiative detachment was observed, for L-mode conventional divertor discharges, to evolve in MAST-U in the same fashion as in a large aspect ratio tokamak. The inner target detaches first followed by the outer divertor. Once the radiation peak reaches the x-point, rises along the inner separatrix and can then evolve into a high field side midplane MARFE-like structure.

This particular implementation of the IRVB did not allow for the reconstruction of the radiation profile in the SXD chamber, but it is nevertheless useful for estimating the total radiated losses in that volume. For a super-x divertor shot the change in integrated radiated power in the SXD chamber and in the rest of the outer leg (below the x-point) were compared with changes in MWI-derived measures of ionization and recombination and radiated power estimates derived from spectroscopy. The radiation in the SXD chamber is observed to decrease in correlation with the movement of the ionisation front (end of the Fulcher band emitting region towards the target) out of the SXD chamber, while in the rest of the outer leg (between SXD chamner entrance and x-point) it increases. This indicates that most of the plasma radiative losses are hydrogenic. This is in contrast with another device of similar size, TCV, where during detachment most of the radiation is due to carbon (intrinsic impurity in both tokamaks).

It was also inferred that between the detachment phases where ionisation and then EIR dominate, there is another phase where plasma interactions with molecular precursors like ${H_2}^+$ and $H^-$ become important. This points to the relevance of these interactions and the need to clarify their role in tokamaks.


\section{Magnum-PSI}

The activities on Magnum-PSI have been dedicated to understanding the ELM burn through process in a detached divertor. Magnum-PSI is a linear device, so it does not compare exactly with a tokamak divertor. However, diagnostic access is much better and it is far easier to consistently repeat the same conditions to accumulate data.

It was found that, indeed, it is possible to fully dissipate in the volume the energy in ELM-like pulses, meaning that essentially no measurable fraction of the ELM-like pulse energy reaches target. This was achieved with pulses with an ELM heat flux factor comparable to what is measured in current tokamaks like JET, albeit much smaller than is expected in ITER. Considering that , in ITER, a much longer connection length is possible between the target and the edge of the ionisation front compared to the Magnum-PSI target chamber size, and the similar neutral pressures, it might be possible to significantly reduce the ELM energy to the target via deep detachment. This has to be balanced with the potentially negative effects of deep detachment on the core plasma. The connection length and the neutral density behind the ionisation front can be increased with a limited effect on the core with baffling and advanced divertor configurations like the super-x. All of that can be tested in MAST-U.

Interpreting the data from an upgraded optical emission spectrograph via a Bayesian framework, it was found that radiative losses in Magnum-PSI are important in reducing the plasma energy ($\sim$1/3 of the ELM-like pulse energy input), but the dominant energy losses are due to potential energy exchanges. Ionisation is dominant for a hot ($T_e>$5eV) plasma, molecular processes are important at lower temperatures and electron ion recombination becomes dominant below 1.5eV.

Molecular assisted dissociation is estimated to be the reaction path, through molecular precursors ${H_2}^+$ and $H^-$, with the biggest impact to the particle and power balance, being more efficient than ${H_2}$ electron impact dissociation. This causes potential energy losses, limiting what is available for ionisation, in a process analogous to "power limitation" in tokamaks.

If detailed simulations of the ELM burn through are needed, it is necessary to include molecular reactions, especially if deeply detached conditions are of interest, as they can provide a significant reduction of the heat flux to the target. The understanding gained from the analysis of the simulated ELM burn through in linear devices, like Magnum-PSI, can help to improve the predictive capability of the fraction of ELM energy that can be removed before it reaches the target.


\section{Outlook}\label{Outlook}


The design, calibration and verification of the IRVB diagnostic, still classified as a prototype in MAST-U, has allowed best practices and improvements for future implementations to be found, such as:

\begin{itemize}
    \item Orient the camera and viewport such to avoid self reflection
    \item Perform the temperature calibration of the IR camera with a large black body source encompassing the whole FOV
    \item Use a foil with the coating applied via chemical vapour deposition rather than sprayed to reduce its non uniformity
    \item Aim the FOV in the co-NBI direction to prevent neutrals from charge exchange to enter through the pinhole and effect measurements
\end{itemize}

Given the lower than expected noise level observed in MU01, the IRVB geometry was modified ahead of MU02 to concentrate the FOV even more around the x-point and divertor. This should provide an even greater spatial resolution, allowing a better assessment of the penetration of the radiator inside the core and possibly the radiation distribution in the SXD chamber.

It is also planned, by the ORNL collaborators with MAST-U, to add a new IRVB aimed at the upper x-point. Together with an additional resistive bolometry system for the upper divertor and the refinement of the current core and lower divertor resistive systems, this will allow monitoring of the radiated power profile in the entire machine. This is necessary for power balance studies and will allow a full characterisation of the evolution of radiation with detachment for all divertor configurations possible in MAST-U.

Further improvements will also come from better data analysis, like implementing the Empirical Bayes method for the simultaneous estimation of regularisation and negative correction coefficients. Another avenue is to try merging the resistive system and IRVBs to perform a tomographic inversion of the entirety of the MAST-U plasma. This could return even more accurate estimates of the integrated radiated power.

Regarding the study of ELM burn through, implementing a spectroscopic setup capable of measuring intra-ELM line emission in tokamaks is very challenging, so a better understanding could still come from linear machines. The crude plasma column model detailed in this work could be improved allowing a gradual change of the plasma properties along the plasma column. This could be achieved by defining an analytical radial and axial (in $z$ and $r$) profile shape for parameters like $n_e$, $T_e$, $n_{H_2}$, $n_{H}$, $n_{{H_2}^+}/n_{H_2}$, $n_{{H}^-}/n_{H_2}$ and then finding the most likely values for the parameters of the analytical expressions in a similar fashion as done in \cite{Gahle2020}. This could be coupled with a model for the neutrals capable of accounting for transport such as Eunomia. Additionally, other experimental data like the total radiated power from bolometry and the flow velocity from collective Thomson scattering (which was later added to the Magnum-PSI diagnostic suite) could be added to the Bayesian analysis.

It might also be important to repeat the experiment with extrinsic impurities, to establish the impact on the radiated power losses and if they facilitate the ELM baffling.

